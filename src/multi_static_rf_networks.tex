\renewcommand{\baselinestretch}{2} \small\normalsize
\chapter{Introduction and Background}
Analysis of RADAR systems in maritime environments is complicated by the fact that the ocean does not generally provide a smooth or uniform surface to work with. Altitude variations change the aspect angle for multipath bounces, induce wave blockage, and add clutter and spikes to the echo return \cite{skolnik_handbook}, \cite{blake_radar}, \cite{nathanson_radar}. Understanding the impact of the sea surface on propagation is critical to evaluating the performance of a RADAR system in a maritime environment.

\section{Multistatic RF Sensor Network Concept}
When we look at cases where the transmitter and receiver are not colocated, we have a bistatic system and the problem becomes even more difficult \cite{willis_bistatic}. With dditional receivers or transmitters, the problem becomes multistatic with multiple bistatic elements. Of particular concern for this document is the case with a single transmitter and multiple receivers.

An example multistatic RF sensor network in a maritime environment is shown in Figure \ref{ms_fig:1}. In this concept, a single transmitter illuminates a target and the echo signal is captured by a pair of receivers. The received signal will fluctuate due to multipath reflections from the surface, path variations, and relative motion of the target. In order to determine the probability of the target being detected by either receiver, we need to understand the statistics of the received signals.

\begin{figure}[H]
  \begin{center}
\includegraphics[width=5in]{../media/multistatic/ms_rf_concept.png}
  \end{center}
  \renewcommand{\baselinestretch}{1} \small\normalsize
  \begin{quote}
    \caption[Multistatic RF Sensor Networks Concept]{Multistatic RF Sensor Networks Concept\label{ms_fig:1}}
  \end{quote}
\end{figure}
\renewcommand{\baselinestretch}{2} \small\normalsize

\section{Probability of Detection}

\section{RADAR Range Equation} 