\renewcommand{\baselinestretch}{2} \small\normalsize
\chapter{Propagation Statistics}
This chapter describes the propagation factor statistics after numerically propagating over independent random sea surface realizations in a Monte Carlo fashion. First we will discuss sampling constraints needed to capture the statistics appropriately and then look at the Monte Carlo run setup and provide some example data before finally fitting to the PDFs.

\section{Sampling Constraints}
The initial set of runs was performed at Ka-Band (35 GHz) and the mean and standard deviation for a 100 run Monte Carlo set are shown in Figure \ref{stat_fig:1}. This data set took over 48 hours to complete running on a laptop and the results were washed out at near range (Figure \ref{stat_fig:1} is clipped at 10 km in near range).

\begin{figure}[H]
  \begin{center}
\includegraphics[width=5in]{../media/statistics/ka_band_stats.png}
  \end{center}
  \renewcommand{\baselinestretch}{1} \small\normalsize
  \begin{quote}
    \caption[Ensemble Statistics at Ka-Band with Standard Atmosphere]{Ensemble Statistics at Ka-Band with Standard Atmosphere\label{stat_fig:1}}
  \end{quote}
\end{figure}
\renewcommand{\baselinestretch}{2} \small\normalsize

Figure \ref{stat_fig:2} shows the mean and standard deviation for a 100 run Monte Carlo set at X-band (10 GHz). In this case, the near range results are much cleaner and the data set took less than 6 hours to complete.
\begin{figure}[H]
  \begin{center}
\includegraphics[width=5in]{../media/statistics/x_band_stats.png}
  \end{center}
  \renewcommand{\baselinestretch}{1} \small\normalsize
  \begin{quote}
    \caption[Ensemble Statistics at X-Band with Standard Atmosphere]{Ensemble Statistics at X-Band with Standard Atmosphere\label{stat_fig:2}}
  \end{quote}
\end{figure}
\renewcommand{\baselinestretch}{2} \small\normalsize

These results indicate that spatial sampling constraints from numerical propagation are important for obtaining accurate statistics, so we need to understand those impacts. We can look at the phase difference between the two primary paths from the propagation factor for the 2 ray model, $F_p = e^{jkL_1} + \Gamma_1e^{jkL_{so}}$. 

\begin{equation}
\begin{gathered}
\Delta\varphi = k\left[ L_1 - L_{so}\right] \\
\Delta\varphi = -\frac{4\pi h_1h_2}{\lambda L}
\label{stat_eq:1}
\end{gathered}
\end{equation}
\renewcommand{\baselinestretch}{2} \small\normalsize

\noindent The derivative of the phase difference with respect to range is then
\begin{equation}
\frac{d\Delta\varphi}{dL}=\frac{4\pi h_1h_2}{\lambda L^2}
\label{stat_eq:2}
\end{equation}
\renewcommand{\baselinestretch}{2} \small\normalsize

\noindent This equation can be converted from rad/m to rad/sample by multiplying by the spatial sampling distance in range, $\Delta r$. We can insist that this phase shift per sample be smaller than some pre-determined value to provide adequate sampling. It is often convenient to specify a limit in terms of wavelengths and we can enforce the condition that there must be at least $n$ samples per wavelength by letting

\begin{equation}
\frac{4\pi h_1h_2\Delta r}{\lambda L^2} \leq \frac{2\pi \lambda}{n}
\label{stat_eq:3}
\end{equation}

This yields a constraint for the maximum allowable spatial sampling step to ensure $n$ samples per wavelength.
\begin{equation}
\boxed{\Delta r \leq \frac{\lambda^2 L^2}{2nh_1h_2}}
\label{stat_eq:4}
\end{equation}

This sampling constraint is shown in Figure \ref{stat_fig:3} for both the 10 and 35 GHz cases, with $n = 20$ and $\Delta r = 0.5$.

\begin{figure}[H]
  \begin{center}
\includegraphics[width=5in]{../media/statistics/sampling_constraint.png}
  \end{center}
  \renewcommand{\baselinestretch}{1} \small\normalsize
  \begin{quote}
    \caption[Sampling Constraints for Statistical Analysis]{Sampling Constraints for Statistical Analysis\label{stat_fig:3}}
  \end{quote}
\end{figure}
\renewcommand{\baselinestretch}{2} \small\normalsize

\section{Monte Carlo Run Results}
Table \ref{stat_tab:0} shows the inputs that were used in the Monte Carlo runs. The presence of a duct at 20 m should impact the mean value but not the variance or amplitude probability distribution.
\begin{table}[H]
  \begin{center}
      \renewcommand{\baselinestretch}{1} \small\normalsize
  \begin{quote}
    \caption[Propagation Monte Carlo Inputs]{Propagation Monte Carlo Inputs\label{stat_tab:0}}
  \end{quote}
  \begin{tabular} {|c | c |}
    \hline
  \bf{Parameter} & \bf{Value}\\ \hline
  Frequency & 10 GHz \\ \hline
  Transmitter Height & 20 m  \\ \hline
  Antenna Pattern & Sinc  \\ \hline
  Antenna Beamwidth & $8^{\circ}$  \\ \hline
  Maximum Range & 20 km  \\ \hline
  Range Step & 0.5 m  \\ \hline
  Maximum Altitude & 30 m \\ \hline
  Wind Speed & 10 m/s \\ \hline
  Inverse Age Parameter & 0.84 (fully developed) \\ \hline
  Refractivity & Duct at 20 m \\ \hline
  Number of Runs & 500 \\ \hline
\end{tabular}
\end{center}
\end{table}
\renewcommand{\baselinestretch}{2} \small\normalsize


Figure \ref{stat_fig:1a} and Figure \ref{stat_fig:1b} show example propagation factors from two specific sea surface realizations.

\begin{figure}[H]
  \begin{center}
\includegraphics[width=5in]{../media/statistics/pf_1.png}
  \end{center}
  \renewcommand{\baselinestretch}{1} \small\normalsize
  \begin{quote}
    \caption[Example Propagation Factor Realization]{Example Propagation Factor Realization\label{stat_fig:1a}}
  \end{quote}
\end{figure}
\renewcommand{\baselinestretch}{2} \small\normalsize

\begin{figure}[H]
  \begin{center}
\includegraphics[width=5in]{../media/statistics/pf_2.png}
  \end{center}
  \renewcommand{\baselinestretch}{1} \small\normalsize
  \begin{quote}
    \caption[Example Propagation Factor Realization]{Example Propagation Factor Realization\label{stat_fig:1b}}
  \end{quote}
\end{figure}
\renewcommand{\baselinestretch}{2} \small\normalsize

\section{PDF Fitting Results}
As shown in \cite{yeh_first_principles} and \cite{yeh_fading}, we expect the statistics to follow a Rician distribution

\begin{equation}
P = \frac{x}{\sigma^2}\exp\left[\frac{-(x^2 + \nu^2}{2\sigma^2} \right]I_0\left(\frac{x\nu}{\sigma} \right)
\label{stat_eq:5}
\end{equation}
\renewcommand{\baselinestretch}{2} \small\normalsize

\begin{figure}[H]
  \begin{center}
\includegraphics[width=5in]{../media/statistics/constant_range_fit.png}
  \end{center}
  \renewcommand{\baselinestretch}{1} \small\normalsize
  \begin{quote}
    \caption[PDF Fitting at Constant Range]{PDF Fitting at Constant Range\label{stat_fig:4}}
  \end{quote}
\end{figure}
\renewcommand{\baselinestretch}{2} \small\normalsize

\begin{figure}[H]
  \begin{center}
\includegraphics[width=5in]{../media/statistics/constant_altitude_fit.png}
  \end{center}
  \renewcommand{\baselinestretch}{1} \small\normalsize
  \begin{quote}
    \caption[PDF Fitting at Constant Altitude]{PDF Fitting at Constant Altitude\label{stat_fig:5}}
  \end{quote}
\end{figure}
\renewcommand{\baselinestretch}{2} \small\normalsize
