\chapter{Analytical Propagation}
\label{analytical_propagation}

\section{2-Ray Model}
The 2-Ray multipath model over a flat earth is given in Figure \ref{mp_fig:1}. In this figure, $P_1$ and $P_2$ are the two points we are interested in propagating between and $P_2'$ is the mirror image of $P_2$. $L_1$ is then the direct path and $L_{so}$ represents the shortest orbit reflection from the sea surface. The altitudes of the two points from the mean sea surface are $h_1$ and $h_2$, the total downrange distance is $L$, and the distance to the reflection point is $x_m$.

\begin{figure}[H]
  \begin{center}
\includegraphics[width=5in]{../media/analysis/multipath_2_ray.png}
  \end{center}
  \renewcommand{\baselinestretch}{1} \small\normalsize
  \begin{quote}
    \caption[2 Ray Multipath Geometry]{ 2 Ray Multipath Geometry\label{mp_fig:1}}
  \end{quote}
\end{figure}
\renewcommand{\baselinestretch}{2} \small\normalsize

To capture the effects of propagating over a spherical earth, the altitude $h_2$ can be modified with a correction factor \cite{blake_radar} dependent on the effective radius of the earth, $r_e = (4/3) 6,371,000$ m.
\begin{equation}
h_2' = h_2 - \frac{L^2}{2r_e}
\label{mp_eq:0}
\end{equation}

The path lengths are given by geometry and simplified through a binomial expansion.
\begin{equation}
\begin{aligned}
L_1 & = \sqrt{L^2 + (h_1-h_2)^2}  \approx L + \frac{(h_1 - h_2)^2}{2L}\\
L_{so} & = \sqrt{L^2 + (h_1+h_2)^2}  \approx L + \frac{(h_1 + h_2)^2}{2L}\\
\end{aligned}
\label{mp_eq:1}
\end{equation}
\renewcommand{\baselinestretch}{2} \small\normalsize

From Huygen's principle, the propagation factor for the 2-ray model is the sum of plane waves traveling along the paths $L_1$ and $L_{so}$. Here $\Gamma_1$ is the reflection coefficient from the sea surface, $\Gamma_1 = \Gamma(x_m)$.
\begin{equation}
\boxed{F_p = e^{ikL_1} + \Gamma_1e^{ikL_{so}}}
\label{mp_eq:1b}
\end{equation}

We can now look at the phase difference between the two primary paths, $\Delta\varphi = k\left[ L_1 - L_{so}\right]$.
\begin{equation}
\boxed{\Delta\varphi = -\frac{4\pi h_1h_2}{\lambda L}}
\label{mp_eq:7}
\end{equation}
\renewcommand{\baselinestretch}{2} \small\normalsize

\noindent The derivative of the phase difference with respect to range is then
\begin{equation}
\frac{d\Delta\varphi}{dL}=-\frac{4\pi h_1h_2}{\lambda L^2}
\label{mp_eq:8}
\end{equation}

\noindent This can be converted from rad/m to rad/sample by multiplying by the spatial sampling distance in range, $\Delta L$. We can insist that this phase shift per sample be smaller than some pre-determined value to provide adequate sampling. It is often convenient to specify a limit in terms of wavelengths and we can enforce the condition that there must be at least $n$ samples per wavelength by letting

\begin{equation}
\frac{4\pi h_1h_2\Delta L}{\lambda L^2} \leq \frac{2\pi \lambda}{n}
\label{mp_eq:10}
\end{equation}

This yields a constraint for the maximum allowable spatial sampling step to ensure $n$ samples per wavelength.
\begin{equation}
\boxed{\Delta L \leq \frac{\lambda^2 L^2}{2nh_1h_2}}
\label{mp_eq:11}
\end{equation}

\section{Multiple Ray Model}
To include multiple rays, we must consider the region over which the electromagnetic wave reflects from the surface, $\tilde{x}$, as shown in Figure \ref{mp_fig:2}. This figure extends Figure \ref{mp_fig:1}, and  $L_2$ and $L_3$ are the path lengths for the various reflected rays and $s(x)$ is the altitude of the sea surface at $x$. 

\begin{figure}[H]
  \begin{center}
\includegraphics[width=5in]{../media/analysis/multipath_layout.png}
  \end{center}
  \renewcommand{\baselinestretch}{1} \small\normalsize
  \begin{quote}
    \caption[Multipath Geometry for Multiple Ray Analysis]{Multipath Geometry for Multiple Ray Analysis\label{mp_fig:2}}
  \end{quote}
\end{figure}
\renewcommand{\baselinestretch}{2} \small\normalsize

The path lengths  $L_2$ and $L_3$ for the reflected ray can be defined through similar triangles around $x$ and can again be simplified through a binomial expansion. At high altitudes or small sea states, $2h_1s(x) >> s^2(x)$ and we can neglect the $s^2(x)$ term.
\begin{equation}
\begin{aligned}
L_2 &= \sqrt{x^2 + \left( h_1 - s(x)\right)^2}  \approx x + \frac{h_1-s(x))^2}{2x}\\
L_3 & = \sqrt{\left(L - x\right)^2 + \left( h_2 - s(x)\right)^2}  \approx L-x + \frac{h_2 - s(x))^2}{2\left(L-x\right)}\\
\end{aligned}
\label{mp_eq:12}
\end{equation}
\renewcommand{\baselinestretch}{2} \small\normalsize

To solve the propagation problem in the geometry of Figure \ref{mp_fig:2a}, we will find the Green's function for the path from the surface to the receiver and then leverage reciprocity to find the Green's function for the forward path.

\subsection{Green's Function Derivation}
Following Huygen's principle, we can use every point on the sea surface as a source for the reflected wave with the geometry shown in Figure \ref{mp_fig:2a}. Here $\theta$ is defined as the angle between the propagation direction ($\hat{\eta}$) and the direction towards the receiver ($\hat{r}$), and $s$ is the portion of the sea surface that acts as a source. We are assuming here that the region around the reflection point is linear. We are using $\eta$ and $\xi$ as coordinates to prevent confusion with $x$ and $z$ when we introduce them later.

\begin{figure}[H]
  \begin{center}
\includegraphics[width=3in]{../media/analysis/gf_geometry.png}
  \end{center}
  \renewcommand{\baselinestretch}{1} \small\normalsize
  \begin{quote}
    \caption[Green's Function Geometry ]{Green's Function Geometry\label{mp_fig:2a}}
  \end{quote}
\end{figure}
\renewcommand{\baselinestretch}{2} \small\normalsize

The projection of $s$ along $\hat{\eta}$ and $\hat{\xi}$ is $-s\sin\theta$ and $s\cos\theta$ respectively. The Green's function can then be computed through the paraxial wave equation with the source constrained along $s$.

\begin{equation}
2jk_o \frac{\partial G}{\partial \eta} - \frac{\partial^2 G}{\partial \xi^2} = -\delta(\eta + s \sin\theta)\delta(\xi - s \cos\theta)
\label{mp_eq:11a}
\end{equation}

We can apply a coordinate transformation of $\eta' = \eta + s\sin\theta$ and $\xi' = \xi - s\cos\theta$
\begin{equation}
2jk_o \frac{\partial G}{\partial \eta'} - \frac{\partial^2 G}{\partial \xi'^2} = -\delta(\eta')\delta(\xi')
\label{mp_eq:11aa}
\end{equation}

We can then take the Fourier transform of Equation \ref{mp_eq:11aa} along $\xi$

\begin{equation}
2jk_o\frac{\partial \hat{G}}{\partial \eta'} -k^2\hat{G} = -\delta(\eta')
\label{mp_eq:11b}
\end{equation}

Solving away from the delta function yields the following expression for $\hat{G}$:
\begin{equation}
\hat{G}= ce^{-\frac{jk^2}{2k_o}\eta'}
\label{mp_eq:11c}
\end{equation}

To find $c$, we can integrate Equation \ref{mp_eq:11b} over a small region $\pm\epsilon$ and take the limit as $\eta'\rightarrow 0$.
\begin{equation}
\lim_{\eta'\rightarrow 0}\int_{-\epsilon}^{\epsilon}d\eta'\left[2jk_o\frac{\partial\hat{G}}{\partial \eta'}- k^2\hat{G} \right] = -\lim_{\eta'\rightarrow 0}\int_{-\epsilon}^{\epsilon}d\eta'\delta(\eta')\\
\label{mp_eq:11ca}
\end{equation}

$\hat{G}$ must be continuous, so the integral over the $k^2\hat{G}$ term must be 0.
\begin{equation}
\begin{gathered}
\lim_{\eta'\rightarrow 0}2jk_o\int_{-\epsilon}^{\epsilon}d\eta'\frac{\partial\hat{G}}{\partial \eta'}= -1\\
\lim_{\eta'\rightarrow 0}2jk_o\hat{G} = -1 \\
\lim_{\eta'\rightarrow 0}2jk_oce^{-\frac{jk^2}{2k_o}(\eta')} = -1\\
c = \frac{j}{2k_o}\\
\end{gathered}
\label{mp_eq:11cb}
\end{equation}

This yields the final expression for $\hat{G}$:
\begin{equation}
\boxed{\hat{G}= \frac{j}{2k_o}e^{-\frac{jk^2}{2k_o}\eta'}}
\label{mp_eq:11cc}
\end{equation}

To find $G$, we can take the inverse Fourier transform of $\hat{G}$:
\begin{equation}
\begin{aligned}
G &= \mathcal{F}^{-1}\{\hat{G}\} = \frac{j}{2k_o}\frac{1}{2\pi}\int_{-\infty}^{\infty}dk e^{-\frac{jk^2}{2k_o}\eta'}e^{jk\xi'} \\
& = \frac{j}{2k_o}\frac{1}{2\pi}\int_{-\infty}^{\infty}dk e^{-\frac{jk^2}{2k_o}\eta'+jk\xi'} \\
\end{aligned}
\label{mp_eq:11d}
\end{equation}

To solve this, we need to complete the square with respect to $k$
\begin{equation}
\begin{gathered}
-\frac{j\eta'}{2k_o}\left[k^2  -2k\frac{k_o\xi'}{\eta'}\right]\\
-\frac{j\eta'}{2k_o}\left[\left(k  -\frac{k_o\xi'}{\eta'}\right)^2 - \frac{k_o^2\xi'^2}{\eta'^2} \right]\\
-\frac{j\eta'}{2k_o}\left(k  -k_s\right)^2+ \frac{jk_o}{2\eta'}\xi'^2 
\end{gathered}
\label{mp_eq:11e}
\end{equation}

Now we can substitute into Equation \ref{mp_eq:11d}
\begin{equation}
\begin{aligned}
G &= \frac{j}{2k_o}\frac{1}{2\pi}\int_{-\infty}^{\infty}dk e^{-\frac{jx'}{2k_o}\left[\left(k  -k_s\right)^2\right]+ \frac{jk_o}{2\eta'}\xi'^2 } \\
&= \frac{j}{2k_o}\frac{1}{2\pi} \sqrt{\frac{\pi 2k_o}{j\eta'}}e^{j\frac{k_o}{2\eta'}\xi'^2 } \\
&= \frac{j}{2k_o}\sqrt{\frac{k_o}{2\pi j\eta'}}\exp\left[j\frac{k_o\xi'^2}{2\eta'} \right]\\
\end{aligned}
\label{mp_eq:11f}
\end{equation}

Transforming back to $\eta$ and $\xi$ yields
\begin{equation}
\boxed{G =  \frac{j}{2k_o}\sqrt{\frac{k_o}{2\pi j(\eta+s\sin\theta)}}\exp\left[j\frac{k_o(\xi-s\cos\theta)^2}{2(\eta + s\sin\theta)} \right]}
\label{mp_eq:11fa}
\end{equation}

Through reciprocity, the Green's function derived in Equation \ref{mp_eq:11fa} can represent the propagation either to the surface or from the surface. With the appropriate selection of $\eta$, $\xi$, and $\theta$, Green's functions for both the forward and reflected paths can be derived. The full Green's function will then be the product of the Green's functions for the individual paths.

\subsection{Solution of the Wave Equation}
To solve the wave equation, we can make the assumption that the sea surface $s$ is small compared to the altitudes, $h_1$ and $h_2$. This means that $s$ will affect the height of the reflection point but have a negligible impact on the surface normal. Therefore we can integrate over $x$ and apply a Dirichlet boundary condition on $G: G|_s = 0$. Since our Greens' function also satisfies the Sommerfeld radiation condition ($\lim_{r\rightarrow\infty} r\left(\frac{\partial G}{\partial r} -ik_oG\right)= 0$) and the outward surface normal is $-\hat{\xi}$

\begin{equation}
U(\bar{r}) = -\int d\eta U(0) \hat{n} \cdot \nabla G = \int d\eta U(0) \frac{\partial G}{\partial \xi}
\label{mp_eq:12a}
\end{equation}

We can take the derivative with respect to $\xi$ of $G$
\begin{equation}
\begin{aligned}
\frac{\partial G}{\partial \xi} &= \frac{j}{2k_o}\sqrt{\frac{k_o}{j2\pi(\eta+s\sin\theta)}}\frac{jk_o2(\xi-s\cos\theta)}{2(\eta+s\sin\theta)}\exp\left[\frac{jk_o(\xi-s\cos\theta)^2}{2(\eta+s\sin\theta)}\right]\\
&= -\sqrt{\frac{k_o}{j2\pi(\eta+s\sin\theta)}}\frac{(\xi-s\cos\theta)}{2(\eta+s\sin\theta)}\exp\left[\frac{jk_o(\xi-s\cos\theta)^2}{2(\eta+s\sin\theta)}\right]\\
\end{aligned}
\label{mp_eq:12b}
\end{equation}

For the first path, we can let the source at $0$ be a plane wave so that $U(0) = e^{jk_o\eta}$
\begin{equation}
U_1(\bar{r}) =  -\int_{0} ^{x_m} d\eta e^{jk_o\eta} \sqrt{\frac{k_o}{j2\pi(\eta+s\sin\theta)}}\frac{(\xi-s\cos\theta)}{2(\eta+s\sin\theta)}\exp\left[\frac{jk_o(\xi-s\cos\theta)^2}{2(\eta+s\sin\theta)}\right]
\label{mp_eq:12c}
\end{equation}

Now we can simplify this expression. In the paraxial approximation, $\cos\theta \approx 1$ and $\sin\theta \approx 0$ and for the first path, $\xi \rightarrow h_1$ and $\eta\rightarrow x$.
\begin{equation}
U_1(\bar{r}) =  -\int_{0} ^{x_m} dx e^{jk_ox} \sqrt{\frac{k_o}{j2\pi x}}\frac{(h_1-s)}{2x}\exp\left[\frac{jk_o(h_1-s)^2}{2x}\right]
\label{mp_eq:12d}
\end{equation}

To add the second path, we will use $U_1$ as the source and add the reflection coefficient from the surface, $\Gamma$.
\begin{equation}
U(\bar{r}) =  -\int_{x_m}^L d\eta \Gamma U_1e^{jk_o\eta} \sqrt{\frac{k_o}{j2\pi(\eta+s\sin\theta)}}\frac{(\xi-s\cos\theta)}{2(\eta+s\sin\theta)}\exp\left[\frac{jk_o(\xi-s\cos\theta)^2}{2(\eta+s\sin\theta)}\right]
\label{mp_eq:12e}
\end{equation}

As in Equation \ref{mp_eq:12d}, for the second path, $\xi \rightarrow h_2$ and $\eta\rightarrow L-x$.
\begin{equation}
U(\bar{r}) =  -\int_{x_m}^L dx \Gamma U_1 e^{jk_o(L-x)}\sqrt{\frac{k_o}{j2\pi(L-x)}}\frac{(h_2-s)}{2(L-x)}\exp\left[\frac{jk_o(h_2-s)^2}{2(L-x)}\right]
\label{mp_eq:12e}
\end{equation}

Expanding and simplifying yields
\begin{equation}
\begin{gathered}
U(\bar{r}) =  -\int_{x_m}^L dx \Gamma\left(-\int_{0} ^{x_m} dx e^{jk_ox} \sqrt{\frac{k_o}{j2\pi x}}\frac{(h_1-s)}{2x}\exp\left[\frac{jk_o(h_1-s)^2}{2x}\right]\right)  \\e^{jk_o(L-x)}\sqrt{\frac{k_o}{j2\pi(L-x)}}\frac{(h_2-s)}{2(L-x)}\exp\left[\frac{jk_o(h_2-s)^2}{2(L-x)}\right]\\
U(\bar{r}) =  \int_{0}^L dx \Gamma e^{jk_oL} \sqrt{\frac{k_o}{j2\pi x}}\frac{(h_1-s)}{2x}\exp\left[\frac{jk_o(h_1-s)^2}{2x}\right]  \\\sqrt{\frac{k_o}{j2\pi(L-x)}}\frac{(h_2-s)}{2(L-x)}\exp\left[\frac{jk_o(h_2-s)^2}{2(L-x)}\right]\\
\end{gathered}
\label{mp_eq:12f}
\end{equation}

We can use the definitions for $L_2$ and $L_3$ from Equation \ref{mp_eq:12} to reduce this further and provide the final integral for $U(\bar{r})$.
\begin{equation}
\boxed{U(\bar{r}) =  \int_{0}^L dx \Gamma \sqrt{\frac{k_o}{j2\pi x}}\sqrt{\frac{k_o}{j2\pi(L-x)}}\frac{(h_1-s)}{2x}\frac{(h_2-s)}{2(L-x)}e^{jk(L_2+L_3)}}
\label{mp_eq:12g}
\end{equation}

\subsection{Approximations}
The path length for a given reflected ray, $L_r$ is then given by
\begin{equation}
\begin{aligned}
L_r &= L_2 + L_3 \\
& = x + \frac{h_1^2-2h_1s(x)}{2x} +  L-x + \frac{h_2^2 - 2h_2s(x)}{2\left(L-x\right)} \\
& = L + \frac{1}{2}\left[\frac{h_1^2}{x} + \frac{h_2^2}{L-x} \right] - s(x)\left[ \frac{h_1}{x} + \frac{h_2}{L-x}\right] \\
&= L + L_0 - L_s
\end{aligned}
\label{mp_eq:13}
\end{equation}
\renewcommand{\baselinestretch}{2} \small\normalsize

Here $L_0$ represents the deterministic component due to reflection from the surface and $L_s$ represents the random component due to reflection from the surface. The reflection point from the 2-ray model, $x_m$, should be a saddle point and provide the dominant contribution. We can therefore perform a Taylor expansion of $L_0$ about $x_m$.

\begin{equation}
L_0 \approx L_0(x_m) + \frac{1}{2}\frac{d^2L_0}{dx^2}\bigg|_{x_m}(x-x_m)^2
\label{mp_eq:14}
\end{equation}

\begin{equation}
\frac{dL_0}{dx} = \frac{1}{2}\left[\frac{-h_1^2}{x^2} + \frac{h_2^2}{(L-x)^2} \right]
\label{mp_eq:15}
\end{equation}

\begin{equation}
\frac{d^2L_0}{dx^2} = \frac{h_1^2}{x^3} + \frac{h_2^2}{(L-x)^3} 
\label{mp_eq:16}
\end{equation}

Since $\frac{dL_0}{dx}\big|_{x_m} = 0$, we can solve for $x_m$
\begin{equation}
\begin{gathered}
\frac{-h_1^2}{x_m^2} + \frac{h_2^2}{(L-x_m)^2} = 0\\
\frac{-h_1}{x_m} + \frac{h_2}{L-x_m} = 0\\
\frac{h_1}{x_m} = \frac{h_2}{L-x_m}\\
h_1(L-x_m) = h_2x_m\\
x_m = \frac{h_1L}{h_1+h_2}
\end{gathered}
\label{mp_eq:17}
\end{equation}

This gives the following for the lowest Taylor series terms:

\begin{equation}
\begin{aligned}
L_0(x_m) &= \frac{(h_1+h_2)^2}{2L} \\
\frac{d^2L_0}{dx^2}\big|_{x_m}  &= \frac{(h_1+h_2)^4}{h_1h_2L^3} \\
L_s(x_m) &= \frac{2s(x)(h_1 + h_2)}{L}\\
\end{aligned}
\label{mp_eq:17a}
\end{equation}

The expansion of $L_0$ is then
\begin{equation}
L_0 \approx \frac{(h_1+h_2)^2}{2L} + \frac{(h_1+h_2)^4}{2h_1h_2L^3}(x-x_m)^2
\label{mp_eq:18}
\end{equation}

The propagation factor will now include all the reflections from the surface

\begin{equation}
\begin{aligned}
F_p &= e^{ikL_1} + \int_0^Ldx\Gamma  e^{ik(L_2+L_3)}\\
&= e^{ikL_1} + \int_0^Ldx\Gamma  e^{ik(L+L_0-L_s)}\\
&= e^{ikL_1} +  \Gamma_1e^{ik(L+\frac{(h_1+h_2)^2}{2L})}\int_0^Ldx\frac{\Gamma}{\Gamma_1} e^{ik(\frac{L_0''}{2}(x-x_m)^2-L_s)}\\
&= e^{ikL_1} + \Gamma_1e^{ikL_{so}}\int_0^Ldx\frac{\Gamma}{\Gamma_1} \exp\left[\frac{ikL_0''}{2}(x-x_m)^2 - \frac{i2ks(x)(h_1+h_2)}{L}\right]\\
\end{aligned}
\label{mp_eq:20}
\end{equation}

We have a similar expression to Equation \ref{mp_eq:1b} with the propagation factor from the shortest orbit reflection scaled by contributions from all the other reflected rays. letting $x-x_m \rightarrow \tilde{x}$, we can express the propagation factor as

\begin{equation}
\boxed{F_p = e^{ikL_1} + \Gamma_1 e^{ikL_{so}}\int_{-x_m}^{L-x_m}d\tilde{x} \frac{\Gamma(\tilde{x})}{\Gamma_1}\exp\left[\frac{ikL_0''}{2}\tilde{x}^2 - \frac{i2ks(\tilde{x})(h_1+h_2)}{L}\right]}
\label{mp_eq:21}
\end{equation}

\section{Asymptotic Approach for Deterministic Component}
For the deterministic component, we wish to solve the integral given by
\begin{equation}
I = \int_{-x_m}^{L-x_m}d\tilde{x} \frac{\Gamma(\tilde{x})}{\Gamma_1}\exp\left[\frac{ikL_0''}{2}\tilde{x}^2\right]
\label{mp_eq:22}
\end{equation}

The phase of the integrand oscillates rapidly as shown by the real part of the integrand in Figure \ref{mp_fig:3} with a 10m altitude target, Figure \ref{mp_fig:4} with a 20m altitude target and Figure \ref{mp_fig:5} with a 50m altitude target. Each subplot shows the impact of increasing the ground distance, $L$, with the lower right subplot showing the horizon limit.

\begin{figure}[H]
  \begin{center}
\includegraphics[width=4in]{../media/analysis/phaseVariation_30_10}
  \end{center}
  \renewcommand{\baselinestretch}{1} \small\normalsize
  \begin{quote}
    \caption[Real Part of Integrand for $h_1$ = 30m, $h_2$ = 10m]{ Real Part of Integrand for $h_1$ = 30m, $h_2$ = 10m\label{mp_fig:3}}
  \end{quote}
\end{figure}
\renewcommand{\baselinestretch}{2} \small\normalsize

\begin{figure}[H]
  \begin{center}
\includegraphics[width=4in]{../media/analysis/phaseVariation_30_20}
  \end{center}
  \renewcommand{\baselinestretch}{1} \small\normalsize
  \begin{quote}
  \caption[Real Part of Integrand for $h_1$ = 30m, $h_2$ = 20m]{ Real Part of Integrand for $h_1$ = 30m, $h_2$ = 20m\label{mp_fig:4}}
  \end{quote}
\end{figure}
\renewcommand{\baselinestretch}{2} \small\normalsize

\begin{figure}[H]
  \begin{center}
\includegraphics[width=4in]{../media/analysis/phaseVariation_30_50}
  \end{center}
  \renewcommand{\baselinestretch}{1} \small\normalsize
  \begin{quote}
  \caption[Real Part of Integrand for $h_1$ = 30m, $h_2$ = 50m]{ Real Part of Integrand for $h_1$ = 30m, $h_2$ = 10m\label{mp_fig:5}}
  \end{quote}
\end{figure}
\renewcommand{\baselinestretch}{2} \small\normalsize

Even with low altitudes, the phase oscillations will be rapid as $\tilde{x}$ goes past the limits of integration and will cancel out, so we can use the principle of stationary phase and let the limits of integration go to $\pm\infty$. Since the integrand is an even function, we can then take the limit from $0$ to $+\infty$ and multiply by 2.

\begin{equation}
I = \int_{-\infty}^{\infty}d\tilde{x} \frac{\Gamma(\tilde{x})}{\Gamma_1}\exp\left[\frac{ikL_0''}{2}\tilde{x}^2\right] = 2\int_{0}^{\infty}d\tilde{x} \frac{\Gamma(\tilde{x})}{\Gamma_1}\exp\left[\frac{ikL_0''}{2}\tilde{x}^2\right] 
\label{mp_eq:23}
\end{equation}

To solve this equation, we can work in the complex plane as shown in Figure \ref{mp_fig:6}. The contour along the real axis, $\mathcal{C}_1$, is deformed to ensure we do not cross through the singular point at $\tilde{x} = L-x_m$ as given in Equation \ref{mp_eq:13} for the initial expression of $L_0$. From Jordan's Lemma, the contour along the arc $\mathcal{C}_2$ will be $0$. Since there are no singular points enclosed by the contour, the sum of the integrals along the contours is $0$ so that

\begin{equation}
I = -2\int_{\mathcal{C}_3}d\tilde{x} \frac{\Gamma(\tilde{x})}{\Gamma_1}\exp\left[\frac{ikL_0''}{2}\tilde{x}^2\right]  
\label{mp_eq:24}
\end{equation}

\begin{figure}[H]
  \begin{center}
\includegraphics[width=5in]{../media/path_contour-figure0.pdf}
  \end{center}
  \renewcommand{\baselinestretch}{1} \small\normalsize
  \begin{quote}
    \caption[Path Contour]{ Path Contour\label{mp_fig:6}}
  \end{quote}
\end{figure}
\renewcommand{\baselinestretch}{2} \small\normalsize

To ensure we take the path of steepest descent, we can transform $\tilde{x}$ to the complex variable $z$ by applying a phase shift of $\pi/4$. Now we can approximate the integral as

\begin{equation}
\begin{gathered}
I = -2\int_{\mathcal{C}_3}d\tilde{z}e^{i\pi/4} \frac{\Gamma(x_m)}{\Gamma_1}\exp\left[\frac{-kL_0''}{2}\tilde{z}^2\right]  \\
= 2e^{i\pi/4}\int_{0}^{\infty}d\tilde{z}\frac{\Gamma_1}{\Gamma_1}\exp\left[\frac{-kL_0''}{2}\tilde{z}^2\right]  \\
= 2e^{i\pi/4}\sqrt{\frac{2\pi}{kL_o''}}
\end{gathered}
\label{mp_eq:25}
\end{equation}

This yields an asymptotic approximation for the deterministic component of equation \ref{mp_eq:21}.
\begin{equation}
\boxed{F_p = e^{ikL_1} +2\Gamma_1 e^{i\left(kL_{so} + \pi/4\right)}\sqrt{\frac{2\pi}{kL_o''}}}
\label{mp_eq:26}
\end{equation}