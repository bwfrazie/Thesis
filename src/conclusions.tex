\renewcommand{\baselinestretch}{2} \small\normalsize
\section{Conclusions}

TBD

As the bistatic clutter map suggests, wind direction plays a role in the spatial dependence of clutter cross section. Minimized clutter levels are achieved for targets located normal to the wind direction vector with respect to the emitter and reciever nadirs (surface baseline). Interestingly, when this target location concurrently results in a ground azimuth (emitter-target-reciever) angle of zero, the directional clutter minimum is matched for both the reciever and emitter.