\renewcommand{\baselinestretch}{2} \small\normalsize
\section{Conclusions}
As described in detail below, both Aims were addressed through this work.

\subsection{Aim 1}
The results of the Monte Carlo study indicate that RMT allows prediction of the signal statistics in a multi-static configuration, satisfying Aim 1. However, there is still work required to make this a useful technique. 

The numerical results from TEMPER demonstrated a diffractive phase shift in the location and amplitude of multipath peaks and nulls that is dependent on both wind speed and wind direction. Increasing the wind speed uniformly reduces the overall amplitude and drives the phase shift upward in altitude and inward in range. Increasing the wind direction from cross wind to down wind has the same uniform impact on the location of the peaks and nulls, but doesn't uniformly affect the amplitude because changing the wind direction actually changes the shape of the power spectrum.

In addition, the ensemble propagation factor standard deviation shows nulls at both the peaks and nulls of the ensemble mean propagation factor. This makes sense as the effective path length over which there is appreciable reflection is minimized and tends towards a single reflected path that is either directly in phase or directly out of phase.

Finally, the PDFs produced from the Monte Carlo study are Gaussian and are well matched to those from RMT for an equivalent loss parameter. The relationship between the loss parameter, sea state, and overall geometry remains unexplored and would be the key to making this an effective technique. 

\subsection{Aim 2}
The analysis of bistatic clutter and inclusion of the propagation factors satisfies Aim 2. As there are a near infinite number of possible engagement geometries, the discussion provides a roadmap to explore various geometries rather than a definitive answer for all geometries.

As the bistatic clutter map suggests, wind direction plays a role in the spatial dependence of clutter cross section. Minimized clutter levels are achieved for targets located normal to the wind direction vector with respect to the emitter and reciever nadirs (surface baseline). Interestingly, when this target location concurrently results in a ground azimuth (emitter-target-reciever) angle of zero, the directional clutter minimum is matched for both the reciever and emitter.