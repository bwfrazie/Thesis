\chapter{Ocean Surface Modeling}
This chapter discusses the generation of realizations of the sea surface.

\section{Ocean Wave Power Spectra}  \label{os_sec:power_spectra}
In order to generate a random realization of an ocean surface, we first need to understand the power spectrum of the waves. In 1964, the Pierson-Moskowitz (PM) spectrum was developed to relate the power spectrum of the variance of the wave height ($S_{PM}(k)$) to the measured wind speed 19 meters above the sea surface ($U_{19}$) and the coefficient of gravity ($g=9.81 m/s^2$), where $k$ is the wave vector \cite{michel_sea_spectra}.

 \begin{equation}
S_{PM} = \frac{0.0081}{k^3}e^{-0.74\left(\frac{g}{k}\right)^2U_{19}^{-4}}
\label{os_eq:1}
\end{equation}
 \renewcommand{\baselinestretch}{2} \small\normalsize

 The PM power spectrum, $S_{PM}(k)$ is shown in Figure \ref{os_fig:1} and the PM curvature spectrum, $k^3S_{PM}(k)$ is shown in Figure \ref{os_fig:2}. 

 \begin{figure}[H]
  \begin{center}
\includegraphics[width=5in]{../media/PM_variance_spectrum.png}
  \end{center}
  \renewcommand{\baselinestretch}{1} \small\normalsize
  \begin{quote}
    \caption[Pierson Moskowitz Variance Spectrum]{Pierson Moskowitz Variance Spectrum\label{os_fig:1}}
  \end{quote}
\end{figure}
\renewcommand{\baselinestretch}{2} \small\normalsize

\begin{figure}[H]
  \begin{center}
\includegraphics[width=5in]{../media/PM_curvature_spectrum.png}
  \end{center}
  \renewcommand{\baselinestretch}{1} \small\normalsize
  \begin{quote}
    \caption[Pierson Moskowitz Curvature Spectrum]{Pierson Moskowitz Curvature Spectrum\label{os_fig:2}}
  \end{quote}
\end{figure}
\renewcommand{\baselinestretch}{2} \small\normalsize

The PM spectrum has repeatedly been shown to be accurate for gravity waves in fully developed seas, but fails to capture spectral peaks due to high winds over short fetches. This is seen by the flat slope of $S_{PM}(k)$ in Figure \ref{os_fig:1} and Figure \ref{os_fig:2}. Several spectra have been developed to address this limitation and in 1997, Elfouhaily developed a unified spectrum, $S_E(k)$ \cite{elfouhaily}. The Elfouhaily spectrum is broken into low and high frequency regions ($B_l$ and $B_h$).

\begin{equation}
  \label{os_eq:2}
  S_E = k^{-3}\left[B_l + B_h \right]
\end{equation}
\renewcommand{\baselinestretch}{2} \small\normalsize

The low frequency region ($B_l$) is given by Equation \ref{os_eq:3} and the parameters are defined in Table \ref{os_tab:1}.

\begin{equation}
  \label{os_eq:3}
 B_l = \frac{1}{2} \alpha_p \frac{c_p}{c} F_p
\end{equation}
\renewcommand{\baselinestretch}{2} \small\normalsize
\begin{subequations}
   Low frequency spectrum dependencies:
\begin{align*}
  F_p &= L_{PM}J_pe^{-\Omega\left[\sqrt{10}\left(\sqrt{k/_{k_p}} - 1 \right) \right]^{-1}} &  k_p &= g\left(\frac{\Omega}{U_{10}}\right)^2 & c_p &= \frac{U_{10}}{\Omega} \\
   L_{PM} &=e^{-\frac{5}{4}\left(\frac{k}{k_p} \right)^2} &  J_p &= \gamma^\Gamma  & \alpha_p &= 0.006\Omega^{0.55} 
\end{align*}
\end{subequations}
\renewcommand{\baselinestretch}{2} \small\normalsize

\begin{table}[H]
\begin{center}
\begin{tabular} {|c | c |}
  Parameter & Description \\ \hline
  $F_p$ & Long wave side effect function \\ \hline
  $k_p$ &  Wave number of the spectral peak \\ \hline
  $c_p$ &  Phase speed at the spectral peak \\ \hline
  $L_{PM}$ & PM shape spectrum \\ \hline
  $J_p$ & JONSWAP peak enhancement function \\ \hline
   $\alpha_p$ & Generalized Phillips-Kitaigorodskii long wave equilibrium range parameter
\end{tabular}
  \renewcommand{\baselinestretch}{1} \small\normalsize
  \begin{quote}
    \caption[Elfouhaily Low Frequency Spectrum Parameters]{Elfouhaily Low Frequency Spectrum Parameters\label{os_tab:1}}
  \end{quote}
\end{center}
\end{table}
\renewcommand{\baselinestretch}{2} \small\normalsize

The high frequency region ($B_h$) is given by Equation \ref{os_eq:4} and the parameters are defined in Table \ref{os_tab:2}.

\begin{equation}
  \label{os_eq:4}
 B_h = \frac{1}{2} \alpha_m \frac{c_m}{c} F_m
\end{equation}
\renewcommand{\baselinestretch}{2} \small\normalsize
\begin{subequations}
   Low frequency spectrum dependencies:
\begin{align*}
  F_m &= L_{PM}J_pe^{-\frac{1}{4}\left[k/_{k_m} - 1 \right]^2 } & k_m & = 370 \text{ rad/m} &  c_m &=\sqrt{\frac{2g}{k_m}} \\
  u^* &= \sqrt{0.00144}U_{10}  & \alpha_m &= \alpha_o\frac{u^*}{c_m}  \\
  L_{PM} &=e^{-\frac{5}{4}\left(\frac{k}{k_p} \right)^2}  &  J_p &= \gamma^\Gamma
\end{align*}
\end{subequations}
\renewcommand{\baselinestretch}{2} \small\normalsize

\begin{table}[H]
\begin{center}
\begin{tabular} {|c | c |}
  Parameter & Description \\ \hline
  $F_m$ & Short wave side effect function \\ \hline
  $k_m$ &  Wave number at the secondary peak of the curvature spectrum \\ \hline
  $c_m$ &  Minimum phase speed at $k_m$ \\ \hline
  $\alpha_m$ & Generalized Phillips-Kitaigorodskii short wave equilibrium range parameter \\ \hline
  $L_{PM}$ & PM shape spectrum \\ \hline
  $J_p$ & JONSWAP peak enhancement function \\ \hline
\end{tabular}
  \renewcommand{\baselinestretch}{1} \small\normalsize
  \begin{quote}
    \caption[Elfouhaily High Frequency Spectrum Parameters]{Elfouhaily High Frequency Spectrum Parameters\label{os_tab:2}}
  \end{quote}
\end{center}
\end{table}
\renewcommand{\baselinestretch}{2} \small\normalsize

The variables $L_{PM}$ and $J_p$ are found in both $B_l$ and $B_h$ and are given by Equations \ref{os_eq:5}.

\begin{equation}
\begin{gathered}
  \label{os_eq:5}
    \gamma = \begin{cases}
    1.7,& \text{if } 0.84 < \Omega < 1\\
    1.7 + 6\log{\Omega}, & \text{if } 1 < \Omega < 5
  \end{cases} \\
  \Gamma = \exp{\left[- \frac{\left(\sqrt{\frac{k}{kp} - 1} \right)}{2\sigma^2} \right]} \\
  \sigma 0.08\left[1 + 4\Omega^{-3} \right] \\
\end{gathered}
\end{equation}
\renewcommand{\baselinestretch}{2} \small\normalsize

The inverse age parameter ($\Omega$) indicates how developed the sea is; fully developed fro $\Omega = 0.84$, mature for $\Omega = 1.0$, and young for $\Omega > 2.0$. To compare the results, the Elfouhaily and PM variance spectra are shown in Figure \ref{os_fig:3} for wind speeds of 5 m/s and 10 m/s. The inverse age parameter was set to 0.84 to match and the wind speed at 19 m was approximated from $U_{19} \approx 1.p026 U_{10}$. The Elfouhaily spectra are shown by the dashed lines and indicate a deviation from the PM spectra at large wave numbers.

\begin{figure}[H]
  \begin{center}
\includegraphics[width=5in]{../media/elf_vs_PM_variance_spectrum.png}
  \end{center}
  \renewcommand{\baselinestretch}{1} \small\normalsize
  \begin{quote}
    \caption[Elfouhaily vs. Pierson Moskowitz Variance Spectrum]{Elfouhaily vs. Pierson Moskowitz Variance Spectrum\label{os_fig:3}}
  \end{quote}
\end{figure}
\renewcommand{\baselinestretch}{2} \small\normalsize

The Elfohaily and PM curvature spectra are shown in Figure \ref{os_fig:4}, again for wind speeds of 5 m/s and 10 m/s. The secondary peak in the curvature spectra is shown at 370 rad/m.

\begin{figure}[H]
  \begin{center}
\includegraphics[width=5in]{../media/elf_vs_PM_curvature_spectrum.png}
  \end{center}
  \renewcommand{\baselinestretch}{1} \small\normalsize
  \begin{quote}
    \caption[Elfouhaily vs. Pierson Moskowitz Curvature Spectrum]{Elfouhaily vs. Pierson Moskowitz Curvature Spectrum\label{os_fig:4}}
  \end{quote}
\end{figure}
\renewcommand{\baselinestretch}{2} \small\normalsize

To look at the dependence on wind speed, the Elfohaily variance spectrum is shown in Figure \ref{os_fig:5} for wind speeds from 2 m/s to 21 m/s in increments of 2 m/s, matching Figure 8a in \cite{elfouhaily}.

\begin{figure}[H]
  \begin{center}
\includegraphics[width=5in]{../media/elf_variance_spectrum.png}
  \end{center}
  \renewcommand{\baselinestretch}{1} \small\normalsize
  \begin{quote}
    \caption[Elfouhaily Variance Spectrum]{Elfouhaily Variance Spectrum\label{os_fig:5}}
  \end{quote}
\end{figure}
\renewcommand{\baselinestretch}{2} \small\normalsize

The Elfohaily curvature spectrum is shown in Figure \ref{os_fig:6}, again for wind speeds from 2 m/s to 21 m/s in increments of 2 m/s, matching Figure 8b in \cite{elfouhaily}.

\begin{figure}[H]
  \begin{center}
\includegraphics[width=5in]{../media/elf_curvature_spectrum.png}
  \end{center}
  \renewcommand{\baselinestretch}{1} \small\normalsize
  \begin{quote}
    \caption[Elfouhaily Curvature Spectrum]{Elfouhaily Curvature Spectrum\label{os_fig:6}}
  \end{quote}
\end{figure}
\renewcommand{\baselinestretch}{2} \small\normalsize


\section{Generation of Sea Surface Realizations}
\subsection{Sampling Constraints}
In order to generate a random sea surface height realization,  the first thing we need to establish is the number of points to use, $N$, and the spatial domain length to cover, $L$. These parameters define the geometry and sampling of the problem. The spatial domain sampling is given as $\Delta x = \frac{L}{N}$ and the frequency domain sampling is  given as $\Delta k = \frac{2\pi}{L}$.

The highest wave number that can be sampled given these values is $k_{max} = \frac{N}{2}\Delta k = \frac{N\pi}{L} = \frac{\pi}{\Delta x}$ rad/m, which corresponds to a minimum wavelength of $\lambda_{min} = 2\Delta x = \frac{2L}{N} = \frac{4\pi}{\Delta k N}$ m.

\subsection{Power Spectrum Discretization}

We next follow the procedure outlined in \cite{percival_spectra},  where we work in the frequency domain and start with a set of Gaussian distributed random variables, $W_j$. We need to scale these random numbers by the square root of the Elfouhaily variance spectrum, arrange them so the sequence is Hermitian, and then take the inverse Fourier transform. For a sequence of $N$ variables, and a two-sided continuous power spectrum, $S(k)$, we can generate the frequency domain random variables, $V(k)$ as shown in Equation \ref{os_eq:6}.

\begin{equation}
  \label{os_eq:6}   
  V_j = \begin{cases}
    \sqrt{S_0}W_0, & j = 0 \\
    \sqrt{\frac{1}{2}S_j}\left(W_{2j-1} + iW_{2j} \right), & 1 \leq j <\frac{N}{2} \\
    \sqrt{S_{N/_2}}W_{N-1}, & j = \frac{N}{2} \\
    V_{N-j}^*, &  \frac{N}{2} < j \leq N-1 
  \end{cases} 
\end{equation}

Since we have a one-sided, discrete power spectrum, there are some additional scaling factors that need to be added to Equation \ref{os_eq:6} for conservation of energy. To convert to a discrete power spectrum, we use the fact that the amount of energy contained per unit interval will be a constant, so that $S(f)df = S(k)dk = S(\omega) d\omega$. To normalize the spectrum to the specified wave number sampling interval, we simply need to multiply by $\Delta k$. Next, since we have a one-sided spectrum rather than a two-sided one, we have half the energy and need to divide the power spectrum by 2. With these factors included, we can define the discrete spectrum as $Sd(k) = \frac{S(k)\Delta k}{2}$.

In order to understand the remaining scale factors and evaluate the 2nd moment, we can rewrite Equation \ref{os_eq:6} into a more clear form as shown in Equation \ref{os_eq:7}, letting $u(k) = W_{2-j}$ and $v(k) = W_{2j}$.

\begin{equation}
  \begin{gathered}
  \label{os_eq:7}
  V(+k) = \sqrt{\frac{1}{2}}\left[u(k) + iv(k) \right]\sqrt{S_d(k)} \\
  V(-k) = \sqrt{\frac{1}{2}}\left[u(-k) - iv(-k) \right]\sqrt{S_d(-k)}
  \end{gathered}
\end{equation}

The 2nd moment of either of these functions individually is equal to $S_d(k)$, but the 2nd moment of the combined function is $2S_d(k)$. This means we need to multiply the sequence $V(k)$ by an additional scale factor of $\sqrt{\frac{1}{2}}$ to conserve the total energy. We could also arrive at this result by realizing that in Equation \ref{os_eq:7}, we are starting with a sequence of length $N/2$ and extending it to cover a length of $N$, effectively doubling the energy contained. The scaling factor corrections are shown in Equation \ref{os_eq:8}.

\begin{equation}
  \label{os_eq:8}   
  V_j = \begin{cases}
    \sqrt{\frac{1}{2}S_0\Delta k}W_0, & j = 0 \\
    \sqrt{\frac{1}{2}S_j\Delta k}\frac{1}{2}\left(W_{2j-1} + iW_{2j} \right), & 1 \leq j < \frac{N}{2} \\
    \sqrt{\frac{1}{2}S_{N/_2}\Delta k}W_{N-1}, & j = \frac{N}{2} \\
    V_{N-j}^*, &  \frac{N}{2} < j \leq N-1 
  \end{cases} 
\end{equation}


Now that we have the random Fourier coefficients, we can generate the sea surface through the inverse Fourier transform. Because the sequence is Hermitian, the sea surface will be purely real.

\begin{equation}
  \label{os_eq:9}
  h(x) = \mathcal{F}^{-1}\left\{V(k) \right\}
  \end{equation}

\subsection{Example Realizations}
Figure \ref{os_fig:7} shows one realization of a sea surface with $U_{10}$ = 20 m/s, $N$ = $2^{20}$, and $L$ = 10 km. For these parameters, $\Delta x$ = 9.5 mm, $\Delta k$ = 0.628 mrad/m, $k_{max} = 32.9\times 10^4$ rad/m, and $\lambda_{min} = 19$ mm.

\begin{figure}[H]
  \begin{center}
\includegraphics[width=5in]{../media/realized_sea_surface_20m_per_s.png}
  \end{center}
  \renewcommand{\baselinestretch}{1} \small\normalsize
  \begin{quote}
    \caption[Sea Surface Realization with $U_{10}$ = 20 m/s]{Sea Surface Realization with $U_{10}$ = 20 m/s\label{os_fig:7}}
  \end{quote}
\end{figure}
\renewcommand{\baselinestretch}{2} \small\normalsize

Figure \ref{os_fig:8} shows the original variance spectrum and the locations of the sampling points in wave number space.

\begin{figure}[H]
  \begin{center}
\includegraphics[width=5in]{../media/sampled_spectra_points.png}
  \end{center}
  \renewcommand{\baselinestretch}{1} \small\normalsize
  \begin{quote}
    \caption[Elfouhaily Variance Spectrum Showing Sampled Points]{Elfouhaily Variance Spectrum Showing Sampled Points\label{os_fig:8}}
  \end{quote}
\end{figure}
\renewcommand{\baselinestretch}{2} \small\normalsize

Figure \ref{os_fig:9} shows the original variance spectrum compared to the recovered spectrum that was estimated from the sea surface realization.

\begin{figure}[H]
  \begin{center}
\includegraphics[width=5in]{../media/recreated_test_spectrum.png}
  \end{center}
  \renewcommand{\baselinestretch}{1} \small\normalsize
  \begin{quote}
    \caption[Comparison of Original and Recovered Spectra]{Comparison of Original and Recovered Spectra\label{os_fig:9}}
  \end{quote}
\end{figure}
\renewcommand{\baselinestretch}{2} \small\normalsize
