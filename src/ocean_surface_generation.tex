\chapter{Ocean Surface Modeling}
To capture the physical behavior of electromagnetic waves reflecting off the surface of the ocean, we need to model the surface altitude variations correctly. For this, we first need to understand the power spectra of ocean waves and their dependencies. Next, we can generate random realizations of sea surfaces by discretizing the power spectrum and taking the inverse Fourier transform of a Hermitian sequence of spectrally appropriate coefficients. This will be demonstrated in both the 1-dimensional and 2-dimensional cases.

\section {Waves in the Ocean}
Ocean waves are primarily driven by wind and start when turbulence creates capillary or surface waves with small wavelengths (on the order of a few centimeters). As the wind continues to blow, it increases both the amplitude and the wavelength of the waves until they start to interact. These interactions continue to increase the wavelength and phase speed of the waves until they are traveling faster than the wind, at which point the sea is deemed fully developed. As a sea develops, energy will transfer from higher frequencies to lower frequencies which means the power spectrum will be different for a younger sea than a mature one.

Before discussing the statistical properties of ocean waves, it is useful to review some standard nomenclature. Fetch is the length over which the wind can be considered constant, and a longer fetch generally means higher waves. Significant wave height is the mean wave height of the highest third of the waves. It is represented in the literature as $H_{1/3}$ and is equal to 4 times the standard deviation of the wave height, $H_{1/3} = 4\sigma_h$. Wave height generally refers to the mean difference between successive peaks and troughs (twice the amplitude).

\section{Ocean Wave Power Spectra}  \label{os_sec:power_spectra}
Many methods have been proposed to statistically describe the behavior of ocean waves. These range from the simplest sea state representations with a very coarse binning of wind speed and sea height variations to mathematically complex multi-parameter power spectra.

\subsection {Sea State}
The modern concept of sea state is a combination of the Beaufort wind scale and the Douglas sea scale. The Beaufort wind scale was developed in the early 1800s by sir Francis Beaufort and the Douglas sea scale was developed in the 1920s by H.P Douglas, both of the Royal Navy \cite{uk_met_fact_sheet6}. The Beaufort wind scale measures force vs. wind speed in 12 bins from "Calm" to "Hurricane" and was extended in 1944 to add scales up to 17. In 1960, it was further extended to add probable wave heights. 

The Douglas sea scale on the other hand measures sea height in 9 bins from "Calm (glossy)" to "Phenomenal". The Word Meterological Organization (WMO) has adopted the Douglas sea scale as a standard \cite{wmo_code} and we can compare this to the Beaufort wind scale to add the dependency on wind speed as shown in Table \ref{os_tab:0}. There are no true standards when working with sea states, so care must be taken when comparing results between research groups.

\begin{table}[H]
  \begin{center}
      \renewcommand{\baselinestretch}{1} \small\normalsize
  \begin{quote}
    \caption[WMO Sea State vs. Wind Speed and Wave Height]{WMO Sea State vs. Wind Seed and Wave Height\label{os_tab:0}}
  \end{quote}
  \begin{tabular} {|c | c | c| c|}
    \hline
  \bf{Sea State} & \bf{Descriptio}n & \bf{Wave Height (m)} & \bf{Wind Speed (m/s)}\\ \hline
  0 & Calm (glassy) & 0 & $<$ 0.3 \\ \hline
  1 & Calm (rippled) & 0 - 0.1 & 0.3 - 1.5 \\ \hline
  2 & Smooth (wavelets) & 0.1 - 0.5 & 1.6 - 3.3 \\ \hline
  3 & Slight & 0.5 - 1.25 & 3.4 - 5.5 \\ \hline
  4 & Moderate & 1.25 - 2.5 & 5.5 - 10.7 \\ \hline
  5 & Rough & 2.5 - 4 & 10.8 - 13.8 \\ \hline
  6 & Very Rough & 4 - 6 & 13.9 - 17.1\\ \hline
  7 & High & 6 - 9 & 17.2 - 24.4\\ \hline
  8 & Very High & 9 - 14 & 24.5 - 32.6\\ \hline
  9 & Phenomenal & $\geq$ 14 & $\geq$ 32.7\\ \hline
\end{tabular}
\end{center}
\end{table}
\renewcommand{\baselinestretch}{2} \small\normalsize

\subsection{Pierson-Moskowitz Spectrum}
In 1964, the Pierson-Moskowitz (PM) spectrum given in Equation \ref{os_eq:1} was developed to describe the power spectrum of the variance of wave height in terms of the measured wind speed 19 meters above the sea surface ($U_{19}$) \cite{michel_sea_spectra}. 
 \begin{equation}
S(k) = \frac{0.0081}{k^3}e^{-0.74\left(\frac{g}{k}\right)^2U_{19}^{-4}}
\label{os_eq:1}
\end{equation}
 \renewcommand{\baselinestretch}{2} \small\normalsize
Here, $g$ is the coefficient of gravity ($g = 9.81 m/s^2$). 
 
The PM spectrum is a single parameter spectrum as wind speed only drives the exponential term of Equation \ref{os_eq:1}. The variance power spectrum is shown in the left hand side of Figure \ref{os_fig:1} and the associated curvature spectrum is shown in the right hand side of Figure \ref{os_fig:1}, both for wind speeds from 3 m/s to 21 m/s in increments of 2 m/s. The curvature spectrum is also referred to as the saturation spectrum. It removes the $k^{-3}$ dependence of the power spectrum and allows high spatial frequency behavior to be more readily observed.
 
 \begin{figure}[H]
  \begin{center}
\includegraphics[width=6in]{../media/Ocean_Surface/PM_variance_curvature_spectrum.png}
  \end{center}
  \renewcommand{\baselinestretch}{1} \small\normalsize
  \begin{quote}
    \caption[Pierson-Moskowitz Variance and Curvature Spectra]{Pierson-Moskowitz Variance and Curvature Spectra\label{os_fig:1}}
  \end{quote}
\end{figure}
 \renewcommand{\baselinestretch}{2} \small\normalsize
From Figure \ref{os_fig:1} we can see that wind speed impacts the spectral peak and low spatial frequency cutoff but has no effect at high spatial frequencies.
 
The PM spectrum assumes deep water, infinite fetch and no swells and has repeatedly been shown to be accurate for gravity waves in fully developed seas, but fails to capture spectral peaks due to high winds over short fetches. This means that the low spatial frequency component of the power spectrum is valid in almost all cases but the high spatial frequency component is not.

\subsection{Bretschneider Spectrum}
To overcome the limitations of a spectrum that requires fully developed seas, several multi-parameter spectra have been developed. In 1978, the 12th International Towing Tank Conference recommended the Bretschneider spectrum \cite{michel_sea_spectra} given in Equation \ref{os_eq:1a} where the modal frequency $\omega_m$ is given by $\omega_m = 0.4\sqrt{g/H_{1/3}}$.
\begin{equation}
  \begin{gathered}
  \label{os_eq:1a}
  S(k) = \frac{1.25 \omega_m^4}{8k^3g^2}e^{-1.25\frac{\omega_m^4}{g^2k^2}} 
  \end{gathered}
\end{equation}
\renewcommand{\baselinestretch}{2} \small\normalsize
The variance power spectrum is shown in the left hand side of Figure \ref{os_fig:1a} and the associated curvature spectrum is shown in the right hand side of Figure \ref{os_fig:1a}, both for wave height standard deviations from 0.1 m to 1 m in increments of 0.1 m.

 \begin{figure}[H]
  \begin{center}
\includegraphics[width=6in]{../media/Ocean_Surface/bs_variance_curvature_spectrum.png}
  \end{center}
  \renewcommand{\baselinestretch}{1} \small\normalsize
  \begin{quote}
    \caption[Bretschneider Variance and Curvature Spectra]{Bretschneider Variance and Curvature Spectra\label{os_fig:1a}}
  \end{quote}
\end{figure}
 \renewcommand{\baselinestretch}{2} \small\normalsize
The Bretschneider spectrum is a two-parameter spectrum as both the exponential term and the amplitude in Equation \ref{os_eq:1a} are dependent on significant wave height.

\subsection{JONSWAP Spectrum}
The Joint North Sea Wave Project (JONSWAP) in the 1970s modified the Pierson-Moskowitz spectrum by adding a peak enhancement factor as shown in Equation \ref{os_eq:1b} \cite{michel_sea_spectra}.
\begin{equation}
  \label{os_eq:1b}
  S(k) = S_{PM}(k)\nu^{\frac{k-k_0}{2\sigma^2k_0}} 
  \end{equation}
The enhancement factor $\nu$ was empirically derived for locations in the North Sea and allows control of the amplitude as in the Bretschneider spectrum. The difference is that the JONSWAP enhancement factor only amplifies the spectrum around the spectral peak.

\subsection{Elfouhaily Spectrum}
Many additional spectra have been developed and in 1997, Elfouhaily et. al extended several of these to provide a unified modern spectrum that is broken into low and high spatial frequency regions, $B_l$ and $B_h$, \cite{elfouhaily}. 

\subsubsection{Spectrum Definition}
The 1-dimensional version of the Elfouhaily spectrum is given in Equation \ref{os_eq:2}.
\begin{equation}
  \label{os_eq:2}
  S(k) = k^{-3}\left[B_l + B_h \right]
\end{equation}
\renewcommand{\baselinestretch}{2} \small\normalsize
This spectrum is dependent on the wind speed at 10 m altitude ($U_{10}$) and the inverse age parameter ($\Omega$). The inverse age parameter indicates how developed the sea is; fully developed for $\Omega = 0.84$, mature for $\Omega = 1.0$, and young for $\Omega > 2.0$. 

The low spatial frequency region, $B_l$, is given by Equation \ref{os_eq:3} and the parameters are defined in Table \ref{os_tab:1} and Equation \ref{os_eq:3a}. Here we are using $v$ to represent phase speed rather than $c$ to prevent confusion with the speed of light.
\begin{equation}
  \label{os_eq:3}
 B_l = \frac{1}{2} \alpha_p \frac{v_p}{v} F_p
\end{equation}
\renewcommand{\baselinestretch}{2} \small\normalsize
\begin{subequations}
\label{os_eq:3a}
   Low spatial frequency spectrum dependencies:
\begin{align}
  F_p &= L_{PM}J_pe^{-\frac{\Omega}{\sqrt{10}}\left[\sqrt{k/_{k_p}} - 1 \right]} &  k_p &= g\left(\frac{\Omega}{U_{10}}\right)^2 & v_p &= \frac{U_{10}}{\Omega} \\
   L_{PM} &=e^{-\frac{5}{4}\left(\frac{k_p}{k} \right)^2} &  J_p &= \gamma^\Gamma  & \alpha_p &= 0.006\Omega^{0.55} 
\end{align}
\end{subequations}
\renewcommand{\baselinestretch}{2} \small\normalsize
\begin{table}[H]
  \begin{center}
      \renewcommand{\baselinestretch}{1} \small\normalsize
  \begin{quote}
    \caption[Elfouhaily Low Spatial Frequency Spectrum Parameters]{Elfouhaily Low Spatial Frequency Spectrum Parameters\label{os_tab:1}}
  \end{quote}
  \begin{tabular} {|c | c |}
    \hline
  \bf{Parameter} & \bf{Description} \\ \hline
  $F_p$ & Long wave side effect function \\ \hline
  $k_p$ &  Wave number of the spectral peak \\ \hline
  $v_p$ &  Phase speed at the spectral peak \\ \hline
  $L_{PM}$ & PM shape spectrum \\ \hline
  $J_p$ & JONSWAP peak enhancement function \\ \hline
  $\alpha_p$ & Generalized Phillips-Kitaigorodskii long wave equilibrium range parameter\\ \hline
  $v$ & Phase speed of the wave \\ \hline
\end{tabular}
\end{center}
\end{table}
\renewcommand{\baselinestretch}{2} \small\normalsize
The high spatial frequency region, $B_h$, is given by Equation \ref{os_eq:4} and the parameters are defined in Table \ref{os_tab:2} and Equations \ref{os_eq:4a} and \ref{os_eq:4b}.
\begin{equation}
  \label{os_eq:4}
 B_h = \frac{1}{2} \alpha_m \frac{v_m}{v} F_m
\end{equation}
\renewcommand{\baselinestretch}{2} \small\normalsize
\begin{subequations}
\label{os_eq:4a}
   High spatial frequency spectrum dependencies:
\begin{align}
  F_m &= L_{PM}J_pe^{-\frac{1}{4}\left[k/_{k_m} - 1 \right]^2 } & k_m & = 370 \text{ rad/m} &  v_m &=\sqrt{\frac{2g}{k_m}} = 0.23 \text{ m/s} \\
  u^* &= \sqrt{Cd_{10N}}U_{10}  & L_{PM} &=e^{-\frac{5}{4}\left(\frac{k_p}{k} \right)^2}  &  J_p &= \gamma^\Gamma
\end{align}
\end{subequations}
\renewcommand{\baselinestretch}{2} \small\normalsize
\begin{equation}
\begin{gathered}
  \label{os_eq:4b}
   \alpha_m= \begin{cases}
    10^{-2}\left[1 + \log\left(\frac{u^*}{v_m} \right) \right],& \text{if } u^* \leq v_m\\
    10^{-2}\left[1 + 3\log\left(\frac{u^*}{v_m} \right) \right], & \text{if } u^* > v_m\\
  \end{cases}
\end{gathered}
\end{equation}
\renewcommand{\baselinestretch}{2} \small\normalsize

\begin{table}[H]
  \begin{center}
      \renewcommand{\baselinestretch}{1} \small\normalsize
  \begin{quote}
    \caption[Elfouhaily High Spatial Frequency Spectrum Parameters]{Elfouhaily High Spatial Frequency Spectrum Parameters\label{os_tab:2}}
  \end{quote}
  \begin{tabular} {|c | c |}
    \hline
  \bf{Parameter} & \bf{Description} \\ \hline
  $F_m$ & Short wave side effect function \\ \hline
  $k_m$ &  Wave number at the secondary peak of the curvature spectrum \\ \hline
  $v_m$ &  Minimum phase speed at $k_m$ \\ \hline
  $\alpha_m$ & Generalized Phillips-Kitaigorodskii short wave equilibrium range parameter \\ \hline
  $L_{PM}$ & PM shape spectrum \\ \hline
  $J_p$ & JONSWAP peak enhancement function \\ \hline
  $Cd_{10N}$ & Neutral stability drag coefficient at 10 m above sea level, $\approx 0.00144$ \\ \hline
  $v$ & Phase speed of the wave \\ \hline
\end{tabular}
\end{center}
\end{table}
\renewcommand{\baselinestretch}{2} \small\normalsize
The variables $L_{PM}$ and $J_p$ are found in both $B_l$ and $B_h$ and are given by Equation \ref{os_eq:5}.
\begin{equation}
\begin{gathered}
  \label{os_eq:5}
    \gamma = \begin{cases}
    1.7,& \text{if } 0.84 < \Omega < 1\\
    1.7 + 6\log{\Omega}, & \text{if } 1 < \Omega < 5
  \end{cases} \\
  \Gamma = \exp{\left[- \frac{\left(\sqrt{\frac{k}{kp} - 1} \right)^2}{2\sigma^2} \right]} \\
  \sigma = 0.08\left[1 + 4\Omega^{-3} \right] \\
\end{gathered}
\end{equation}
\renewcommand{\baselinestretch}{2} \small\normalsize
In \cite{elfouhaily}, the dispersion relation that holds for both gravity and capillary waves is given as 
\begin{equation}
\label{os_eq:5aa}
\omega^2 = gk\left[1 + \left(\frac{k}{k_m}\right)^2 \right]
\end{equation}
We can use Equation \ref{os_eq:5aa} to define the phase speed as
\begin{equation}
\label{os_eq:5ab}
v = \frac{\omega}{k}= \sqrt{\frac{g}{k}\left[1 + \left(\frac{k}{k_m}\right)^2 \right]}
\end{equation}

In the case where we have fetch rather than the inverse age parameter, we can compute the inverse age parameter as shown in Equation \ref{os_eq:5a}. Here $x$ is the dimensional fetch in m and $X$ is the non-dimensional fetch.
\begin{equation}
\label{os_eq:5a}
\begin{gathered}
 X = \frac{g}{U_{10}^2}x\\
 X_0 = 2.2 \times 10^4 \\
 \Omega = 0.84\tanh\left[\left(\frac{X}{X_0} \right)^{0.4} \right]^{-0.75} \\
\end{gathered}
\end{equation}
\renewcommand{\baselinestretch}{2} \small\normalsize

\subsubsection{Spectrum Visualization}
The variance power spectrum is shown in the left hand side of Figure \ref{os_fig:3} and the associated curvature spectrum is shown in the right hand side of Figure \ref{os_fig:3}. These figures were generated for wind speeds from 3 m/s to 21 m/s in increments of 2 m/s, matching Figure 8 in \cite{elfouhaily}. In these figures, the secondary peak can clearly be seen at 370 rad/m.
\begin{figure}[H]
  \begin{center}
\includegraphics[width=6in]{../media/Ocean_Surface/elf_variance_curvature_spectrum.png}
  \end{center}
  \renewcommand{\baselinestretch}{1} \small\normalsize
  \begin{quote}
    \caption[Elfouhaily Variance and Curvature Spectra vs. $U_{10}$]{Elfouhaily Variance and Curvature Spectra vs. $U_{10}$\label{os_fig:3}}
  \end{quote}
\end{figure}
\renewcommand{\baselinestretch}{2} \small\normalsize

To demonstrate the impact from the inverse age parameter, the variance power spectrum is shown in the left hand side of Figure \ref{os_fig:3a} and the associated curvature spectrum is shown in the right hand side of Figure \ref{os_fig:3a}. These figures were generated for $\Omega$ ranging from $0.84$ (fully developed) to $5.0$ (very young). From this figure, we can see that the spectra show energy transferring from high spatial frequencies to low spatial frequencies and the spectral peak flattening as the sea develops.
\begin{figure}[H]
  \begin{center}
\includegraphics[width=6in]{../media/Ocean_Surface/elf_variance_curvature_spectrum_age.png}
  \end{center}
  \renewcommand{\baselinestretch}{1} \small\normalsize
  \begin{quote}
    \caption[Elfouhaily Variance and Curvature Spectra vs. $\Omega$]{Elfouhaily Variance and Curvature Spectra vs. $\Omega$ \label{os_fig:3a}}
  \end{quote}
\end{figure}
\renewcommand{\baselinestretch}{2} \small\normalsize

\subsubsection{Comparison with PM Spectrum}
To compare the results, the Elfouhaily and PM variance power spectra are shown in the left hand side of Figure \ref{os_fig:2} for wind speeds of 5 m/s and 10 m/s. The inverse age parameter was set to 0.84 to match and the wind speed at 19 m altitude was approximated as $U_{19} \approx 1.026 U_{10}$. The Elfouhaily spectra are shown by the dashed lines and indicate a deviation from the PM spectra at large wave numbers. The corresponding curvature spectra are shown in the right hand side of Figure \ref{os_fig:2}. 
\begin{figure}[H]
  \begin{center}
\includegraphics[width=6in]{../media/Ocean_Surface/elf_vs_PM_variance_curvature_spectrum.png}
  \end{center}
  \renewcommand{\baselinestretch}{1} \small\normalsize
  \begin{quote}
    \caption[Elfouhaily vs. Pierson Moskowitz Variance and Curvature Spectra]{Elfouhaily vs. Pierson Moskowitz Variance and Curvature Spectra\label{os_fig:2}}
  \end{quote}
\end{figure}
\renewcommand{\baselinestretch}{2} \small\normalsize

\subsection{Directional Spreading Functions}
All the variance power spectra that have been discussed are uni-directional, meaning they only apply downrange in the direction of the wind. For two-dimensional surfaces, we need to include angular extent. This can be accomplished by applying a spreading function $\Phi$ in $k$-space as shown in Equation \ref{os_eq:5b} \cite{elfouhaily}.
\begin{equation}
\label{os_eq:5b}
\Psi(k,\phi) = \frac{1}{k}S(k)\Phi(k,\phi)
\end{equation}
\renewcommand{\baselinestretch}{2} \small\normalsize
As discussed in \cite{elfouhaily}, the spreading function should be symmetric in upwind-crosswind and will have the form given in Equation \ref{os_eq:5c}.
\begin{equation}
\label{os_eq:5c}
\Psi(k,\phi) = \frac{1}{2\pi}\left[1 + \Delta(k)\cos(2\phi) \right]
\end{equation}
\renewcommand{\baselinestretch}{2} \small\normalsize

Here $\Delta(k)$ is defined as in Equation \ref{os_eq:5d}.
\begin{equation}
\label{os_eq:5d}
\begin{gathered}
\Delta(k) = \tanh\left( a_0 + a_p\left(\frac{v}{v_p}\right)^{2.5}  + a_m\left(\frac{v_m}{v} \right)^{2.5}\right)\\
a_0 = \frac{\log(2)}{4} \\
a_p = 4\\
a_m = 0.13\frac{u^*}{v_m} \\
\end{gathered}
\end{equation}
\renewcommand{\baselinestretch}{2} \small\normalsize

The complete 2-dimensional unified Elfouhaily spectrum is given in Equation \ref{os_eq:5e}.
\begin{equation}
\label{os_eq:5e}
\boxed{\Psi(k,\phi) = \frac{1}{2\pi k^4}\left[B_l + B_h \right] \left[1 + \Delta(k)\cos(2\phi) \right]}
\end{equation}

\section{Generation of Sea Surface Realizations in 1-Dimension}\label{os_sec:1d}
Realizations of sea surfaces rely on having adequate sampling to cover both the spectral and spatial ranges of interest and then generating an appropriate set of Fourier coefficients. The 1-dimensional case is easiest to both understand and implement and will be discussed first.

\subsection{Sampling Constraints}\label{os_label:1d_sampling_constraints}
In order to generate a random sea surface, we need to establish the number of points to use ($N$) and the spatial domain length to cover ($L$). The spatial domain sampling inteval is then $\Delta x = \frac{L}{N}$ and the frequency domain sampling interval is $\Delta k = \frac{2\pi}{L}$. 

From the Nyquist sampling theorem, the highest wave number that can be sampled given these values is $k_{max} = \frac{N}{2}\Delta k = \frac{N\pi}{L}$ rad/m, which corresponds to a minimum wavelength of $\lambda_{min} = 2\Delta x = \frac{2L}{N}$ m.

Immediately, we see that to cover up to the secondary peak at $k_m$, $\frac{N}{L}\pi > k_m$ or $N > \frac{k_m}{\pi}L$. Since $k_m = 370$ rad/m, we can round up to state that $N \geq 118L$. In other words, the number of points required to cover up to the secondary peak in the spectrum is 118 times more than the length of the spatial domain. For the 1-dimensional case, this is not too bad as a 10 km length can be sampled with 1,180,000 points. It is in the 2-dimensional case where this really becomes problematic, as we now need $N^2$ points.

Figure \ref{os_fig:6aa} shows the impact on wavenumber sampling for various values of $N$ for $L$ = 1 km. The upper left hand plot shows the case where $N = 5L$, the upper right hand plot shows the case where $N = 10L$, the lower left hand plot shows the case where $N = 118L$, and the lower right hand plot shows the case where $N = 500L$. Since $\Delta x = \frac{L}{N}$, these cases represent spatial resolutions of 1/5 m, 1/10 m, 1/118 m, and 1/500 m. 
\begin{figure}[H]
  \begin{center}
\includegraphics[width=6in]{../media/Ocean_Surface/sampling_coverage.png}
  \end{center}
  \renewcommand{\baselinestretch}{1} \small\normalsize
  \begin{quote}
    \caption[Wavenumber Sampling Coverage]{Wavenumber Sampling Coverage\label{os_fig:6aa}}
  \end{quote}
\end{figure}
\renewcommand{\baselinestretch}{2} \small\normalsize
Since $L$ is the same for each case, $\Delta k$ is also the same and we can see how increasing $N$ increases coverage of higher spatial frequencies.

In many situations, we are not concerned with capturing the high spatial frequencies as much as we are with staying below a pre-defined maximum spatial step. This step is typically on the order of 0.5 m, in which case $N = 2L$.

As a final note for sampling constraints, we need to maintain integer indexing for digital implementation so $N$ must be an even number. Because we will be using an FFT, it is convenient to force $N$ to be a power of 2. We will define the range in $k$-space as running from $-\left(\frac{N}{2}-1\right)\Delta k$ to $\frac{N}{2}\Delta k$.

\subsection{Frequency Domain Representation}
The general idea is to use the power spectrum to create a realization of the sea surface in the frequency domain and then transform that to the spatial domain through an inverse Fourier transform. We will follow the procedure outlined in \cite{percival_spectra}, 
but we need to provide some additional scale factors to account for conservation of energy.

To handle discretization of the power spectrum, we use the fact that the amount of energy contained per unit interval is a constant, so that $S(k)dk = S(\omega) d\omega$. To normalize the spectrum to the specified wave number sampling interval, we simply need to multiply by $\Delta k$. Next, since we have a one-sided spectrum rather than a two-sided one, we only have half the energy present and need to divide the power spectrum by 2. With these factors included, we can define the one-sided, discrete spectrum as $S_d = \frac{S(k)\Delta k}{2}$.

To generate the frequency domain representation ($V_j$), we will take a set of Gaussian distributed random variables that are scaled  by the square root of the power spectrum and arrange them so the sequence is Hermitian, $V(k) = V^*(-k)$. Because $N$ is even, as described in Section \ref{os_label:1d_sampling_constraints}, there are special cases to handle to ensure that the sequence is Hermitian. $V_j$ only contains a single element for $k = 0$ and $k = \frac{N}{2}\Delta k$ and must be purely real at both frequencies. If we take a pair of uncorrelated Gaussian distributed random variables of length $\frac{N}{2}$ ($w$ and $u$), we can generate $V_j$ as shown in Equation \ref{os_eq:8}.

\begin{equation}
  \label{os_eq:8}   
  V_j = \begin{cases}
    \sqrt{\frac{1}{2}S_0\Delta k}w_0, & j = 0 \\
    \frac{1}{2}\sqrt{\frac{1}{2}S_j\Delta k}\left(w_j + iu_j \right), & 1 \leq j < \frac{N}{2} \\
   \sqrt{\frac{1}{2}S_{N/2}\Delta k}u_0 & j = \frac{N}{2} \\
    V_{N-j}^*, &  \frac{N}{2} < j \leq N-1 
  \end{cases} 
\end{equation}
The extra factor of $\frac{1}{2}$ in line 2 of Equation \ref{os_eq:8} comes from the fact that we need to normalize both the expectations $\left<w_j + iu_j\right>$ and $\left<w_j - iu_j\right>$ to conserve energy.

The index ordering is assumed to have $k=0$ at the first element so that the sequence will wrap in frequency at $j = \frac{N}{2} + 1$. An example sequence for $N = 8$ is shown in Table \ref{os_tab:2a}.
\begin{table}[H]
  \begin{center}
      \renewcommand{\baselinestretch}{1} \small\normalsize
  \begin{quote}
    \caption[Example 1-D Hermitian Sequence]{Example 1-D Hermitian Sequence\label{os_tab:2a}}
  \end{quote}
  \begin{tabular} {| c | c | c |}
    \hline
  \bf{$j$} & \bf{$k_j$} & \bf{$V_j$} \\ \hline
  $0$ & $0$ & $\sqrt{\frac{S_{0}}{2} \Delta k}w_0$ \\ \hline
  $1$ & $\Delta k$ & $\frac{1}{2}\sqrt{\frac{S_{1}}{2} \Delta k} \left(w_1 + iu_1 \right)$ \\ \hline
  $2$ & $2\Delta k$ & $\frac{1}{2}\sqrt{\frac{S_{2}}{2} \Delta k} \left(w_2 + iu_2 \right)$ \\ \hline
  $3$ & $3\Delta k$ & $\frac{1}{2}\sqrt{\frac{S_{3}}{2} \Delta k} \left(w_3 + iu_3 \right)$ \\ \hline
  $4$ & $4\Delta k$ & $\sqrt{\frac{S_{4}}{2} \Delta k} u_0$ \\ \hline
  $5$ & $-3\Delta k$ & $\frac{1}{2}\sqrt{\frac{S_{3}}{2} \Delta k} \left(w_3 - iu_3 \right)$ \\ \hline
  $6$ & $-2\Delta k$ & $\frac{1}{2}\sqrt{\frac{S_{2}}{2} \Delta k} \left(w_2 - iu_2 \right)$  \\ \hline
  $7$ & $-\Delta k$ & $\frac{1}{2}\sqrt{\frac{S_{1}}{2} \Delta k} \left(w_1 - iu_1 \right)$ \\ \hline

\end{tabular}
\end{center}
\end{table}
\renewcommand{\baselinestretch}{2} \small\normalsize

Now that we have the random Fourier coefficients, we can generate the sea surface through the inverse Fourier transform as in Equation \ref{os_eq:9}. Because the sequence is Hermitian, the sea surface will be purely real.
\begin{equation}
  \label{os_eq:9}
  h(x) = \mathcal{F}^{-1}\left\{V(k) \right\}
  \end{equation}

\subsection{Example Realizations}
Figure \ref{os_fig:7a} shows an example realization for $U_{10}$ = 10 m/s and $L$ = 1km. The upper plots show the generated sea surface, and a comparison of the original spectrum to the recovered spectrum for $N = 2^{17}$. Here, $N = 131.07L$ and $\Delta x$ = 7.6 mm. The bottom plots show the same for $N=2^{11}$. Here, $N = 2.04L$ and $\Delta x$ = 0.49 m.
\begin{figure}[H]
  \begin{center}
\includegraphics[width=6in]{../media/Ocean_Surface/sea_surface_1000.png}
  \end{center}
  \renewcommand{\baselinestretch}{1} \small\normalsize
  \begin{quote}
    \caption[Example Sea Surface Realization for 1 km Range]{Example Sea Surface Realization for 1 km Range\label{os_fig:7a}}
  \end{quote}
\end{figure}
\renewcommand{\baselinestretch}{2} \small\normalsize

Figure \ref{os_fig:7aa} shows another example realization, this time for $L$ = 10km. The upper plots show the generated sea surface, and a comparison of the original spectrum to the recovered spectrum for $N = 2^{21}$. Here, $N = 209.7L$ and $\Delta x$ = 4.8 mm. The bottom plots show the same for $N=2^{15}$. Here, $N = 3.3L$ and $\Delta x$ = 0.3 m.
\begin{figure}[H]
  \begin{center}
\includegraphics[width=6in]{../media/Ocean_Surface/sea_surface_10000.png}
  \end{center}
  \renewcommand{\baselinestretch}{1} \small\normalsize
  \begin{quote}
    \caption[Example Sea Surface Realization for 10 km Range]{Example Sea Surface Realization for 10 km Range\label{os_fig:7aa}}
  \end{quote}
\end{figure}
\renewcommand{\baselinestretch}{2} \small\normalsize
For both Figures \ref{os_fig:7a} and \ref{os_fig:7aa}, $N$ was set to the next highest power of 2 to capture the secondary peak.

\section{Generation of Sea Surface Realizations in 2-Dimensions}
We will generally follow Section \ref{os_sec:1d} but will now have a wavevector $\mathbf{k} = k_x\hat{x} + k_y\hat{y}$

Figure \ref{os_fig:8} shows a section of a generated 2D sea surface. The total surface was $1000 m^2$, only $100 m^2$ is shown here.
\begin{figure}[H]
  \begin{center}
\includegraphics[width=6in]{../media/Ocean_Surface/sea_surface_2d_surf.png}
  \end{center}
  \renewcommand{\baselinestretch}{1} \small\normalsize
  \begin{quote}
    \caption[Generated 2D Sea Surface, $100 m^2$ Patch]{Generated 2D Sea Surface, $100 m^2$ Patch \label{os_fig:8}}
  \end{quote}
\end{figure}
\renewcommand{\baselinestretch}{2} \small\normalsize

Figure \ref{os_fig:9} shows an image of the entire $1000 m^2$ generated surface.
\begin{figure}[H]
  \begin{center}
\includegraphics[width=6in]{../media/Ocean_Surface/sea_surface_2d_image.png}
  \end{center}
  \renewcommand{\baselinestretch}{1} \small\normalsize
  \begin{quote}
    \caption[Generated 2D Sea Surface]{Generated 2D Sea Surface\label{os_fig:9}}
  \end{quote}
\end{figure}
\renewcommand{\baselinestretch}{2} \small\normalsize

Figure \ref{os_fig:10} shows slices along the $N/2$ element in both the $x$ and $y$ planes along with a comparison of the 1D Elfouhaily spectrum to the spectrum recovered from the slices. The low frequency content does not match as well as in the 1-dimensional case.
\begin{figure}[H]
  \begin{center}
\includegraphics[width=6in]{../media/Ocean_Surface/sea_surface_2d_slices1000.png}
  \end{center}
  \renewcommand{\baselinestretch}{1} \small\normalsize
  \begin{quote}
    \caption[X and Y Slices of 2D Generated Sea Surface]{X and Y Slices of 2D Generated Sea Surface\label{os_fig:10}}
  \end{quote}
\end{figure}
\renewcommand{\baselinestretch}{2} \small\normalsize