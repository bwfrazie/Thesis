\section{Introduction}
This section provides some background on Green’s functions, presents the convention used in this document for a forward propagating wave, discusses how this convention defines the Fourier transform, and introduces the different types of geometric spreading that may be encountered in propagation.

\subsection{Background}
In 1828, an English mathematician named George Green published "An Essay on the Application of Mathematical Analysis to the Theories of Electricity and Magnetism" with a theorem that is now used extensively to solve differential equations in electromagnetics, quantum mechanics, and applied mathematics \cite{green_phys_today}. Unfortunately, the scope of his contributions weren’t realized until after his death. 

Today, Green's functions are a useful tool in solving inhomogeneous differential equations and can be thought of as the impulse response for the differential equation. There are many references for Green's functions from the mathematical perspective \cite{bender_orszag}, \cite{arfken_weber}, \cite{gbur_math}, \cite{guenther_partial_de}, \cite{duffy_green}, and for their application to electromagnetics \cite{jackson_classical_em}, \cite{zangwill_modern_em}, \cite{balanis_advanced}, \cite{goodman_fourier}, \cite{smith_radiation}. Unfortunately, these references typically only highlight one aspect and have conflicting conventions for propagation direction and normalization. In addition, they tend to skip over many steps, and some define Green's functions without really using them, while others use Green’s Functions without really defining them. The purpose of this document is to work through derivations of the canonical free space Green's functions and demonstrate in a consistent manner how they can be utilized in electromagnetic propagation problems where diffraction becomes important. A particular aspect covered here is the derivation of equations governing the propagation of Radio Frequency (RF) waves in a maritime environment.

\subsection{Time Dependence Sign Convention} \label{gf_sec:time_dependence}
Before getting started, we need to define the sign convention used for time dependence. We  assume harmonic propagating waves that follow the typical Engineering literature with time dependence going as $e^{j\omega t}$. This means we can represent an arbitrary vector wave $\mathbf{C}(\mathbf{r},t)$ as:

\begin{equation}
\mathbf{C}\left(\mathbf{r},t\right) = \mathbf{C}\left(\mathbf{r}\right)e^{j\omega t}
\label{gf_eq:0a}
\end{equation}
\renewcommand{\baselinestretch}{2} \small\normalsize

For a system to be causal, it may only depend on past or present inputs and not on future inputs. Any physically realizable system, such as an electromagnetic wave must be causal, as something that has not yet happened cannot have an impact on the current output. We can relate the causality condition to the phase with the restriction that the phase can be delayed but not advanced. This tells us that for propagating waves, we need to look for solutions in the form of $g\left(\pm\left[\omega t - k_or\right]\right)$, as solutions in the form of $g\left(\pm\left[\omega t + k_or\right]\right)$ advance the phase and are not physically realizable. 

With this restriction, our vector waves from Equation \ref{gf_eq:0a} must take the form:

\begin{equation}
\mathbf{C}\left(\mathbf{r},t\right) = \mathbf{C}e^{j\left(\omega t - \mathbf{k}_o \cdot \mathbf{r} \right)}
\label{gf_eq:18b}
\end{equation}
\renewcommand{\baselinestretch}{2} \small\normalsize

In the next section, we will see that this choice of time dependence  defines the forward and reverse Fourier transforms in both the temporal and spatial domains.

\subsection {Fourier Transforms in Time and Space} \label{gf_sec:fourier_transform}
We must remain self consistent with our sign conventions when using Fourier transforms. The easiest way to ensure consistency is to interpret the transform from a frequency domain to either the temporal or spatial domain as a superposition of the signal of interest with forward propagating waves over all frequencies. The sign convention for the Fourier transform then comes directly from our definition of a forward propagating wave from the previous section.  

Table \ref{gf_tab:0a} presents the forward Fourier transforms, $\hat{g} = \mathcal{F}\{g\}$, and the inverser Fourier transforms, $g = \mathcal{F}^{-1}\{\hat{g}\}$, in both time and space for an arbitrary function, $g$, and its Fourier transform, $\hat{g}$.
\begin{table}[ht]
  \begin{center}
      \renewcommand{\baselinestretch}{1} \small\normalsize
  \begin{quote}
    \caption[Fourier Transforms in Time and Space]{Fourier Transforms in Time and Space\label{gf_tab:0a}}
  \end{quote}
  \begin{tabular} {|c | c | c|}
    \hline
  \bf{Domain} & \bf{Forward Transform} & \bf{Inverse Transform}\\ \hline
  Time & $\displaystyle \hat{g}(\omega) = \int_{-\infty}^{\infty} g(t)e^{-j\omega t}dt$ & $\displaystyle  g(t) = \frac{1}{2\pi}\int_{-\infty}^\infty \hat{g}(\omega)e^{j\omega t}d\omega$\\ \hline
 Space & $\displaystyle \hat{g}(k) = \int_{-\infty}^{\infty} g(x)e^{jk x}dx$ & $\displaystyle  g(x) = \frac{1}{2\pi}\int_{-\infty}^\infty \hat{g}(k)e^{-jk x}dk$\\ \hline
\end{tabular}
\end{center}
\end{table}
\renewcommand{\baselinestretch}{2} \small\normalsize

\subsection{Geometric Spreading}
The geometric spreading of an electromagnetic wave is dependent on the dimensionality and approximations used. A plane wave is easy to work with but shows no spreading or amplitude reduction as it propagates. A paraxial wave allows us to represent beam like waves and will have spreading proportional to $1/\sqrt{L_x}$, where $L_x$ is the length along the propagation direction. A cylindrical wave will have spreading proportional to $1/\sqrt{\rho}$ and a spherical wave will have spreading proportional to $1/r$. The amount of geometry induced spreading increases as we add higher fidelity to our models and analytical tools so we need to take that into consideration when comparing models.

\section {Green's Identities and Method}
The two identities named for George Green and the method of Green’s functions will be derived in this section.

\subsection{Identities}
To derive Green’s identities, we will start with the divergence theorem:

\begin{equation}
\oint\limits_{S} \mathbf{C} \cdot \hat{n} dS = \int\limits_{V}d^3r'\nabla \cdot \mathbf{C}
\label{gf_eq:1}
\end{equation}
\renewcommand{\baselinestretch}{2} \small\normalsize

By replacing the vector $\mathbf{C}$ with a function of two arbitrary scalar variables, $\mathbf{C}=\varphi\nabla\psi$, we get:

\begin{equation}
\oint\limits_{S} \varphi\nabla\psi \cdot \hat{n} dS = \int\limits_{V}d^3r'\nabla \cdot \varphi\nabla\psi
\label{gf_eq:2}
\end{equation}
\renewcommand{\baselinestretch}{2} \small\normalsize

Using the identity $\nabla\psi\cdot \hat{n} = \partial \psi/\partial n$ and expanding the dot product in the volume integral yields Green's first identity:

\begin{equation}
\boxed{\oint\limits_{S} \varphi\frac{\partial \psi}{\partial n} dS = \int\limits_{V}d^3r' \left[ \nabla\varphi \cdot \nabla\psi +\varphi \nabla^2 \psi\right]}
\label{gf_eq:3}
\end{equation}
\renewcommand{\baselinestretch}{2} \small\normalsize

If we flip the order of the scalar variables and instead let $\mathbf{C}=\psi\nabla\varphi$, Green's first identity is expressed as:

\begin{equation}
\oint\limits_{S} \psi\frac{\partial \varphi}{\partial n} dS = \int\limits_{V}d^3r' \left[ \nabla\psi \cdot \nabla\varphi +\psi \nabla^2 \varphi\right]
\label{gf_eq:4}
\end{equation}
\renewcommand{\baselinestretch}{2} \small\normalsize

\noindent Subtracting Equation \ref{gf_eq:4} from Equation \ref{gf_eq:3} yields Green's second identity:

\begin{equation}
\boxed{\oint\limits_{S} \left[ \varphi\frac{\partial \psi}{\partial n} - \psi\frac{\partial \varphi}{\partial n} \right]dS = \int\limits_{V}d^3r' \left[ \varphi\nabla^2\psi- \psi \nabla^2 \varphi\right]}
\label{gf_eq:5}
\end{equation}

\subsection {Method of Green's Functions} \label{gf_sec:method}
For the purposes of this document, we are interested in time varying propagation problems. Therefore, we would like to apply the method of Green's functions to the scalar wave equation  to solve for the scalar field $\varphi$ as a function of space and time with an arbitrary driving function, $f$, that is also a function of space and time:

\begin{equation}
\nabla^2\varphi\left(\mathbf{r},t\right) - \frac{1}{c^2}\frac{\partial^2 \varphi\left(\mathbf{r},t\right)}{\partial t^2} = f\left(\mathbf{r},t\right)
\label{gf_eq:6}
\end{equation}
\renewcommand{\baselinestretch}{2} \small\normalsize

\noindent We can utilize the Fourier transform to remove the time dependence:

\begin{equation}
\nabla^2\varphi\left(\mathbf{r},\omega\right) + \frac{\omega^2}{c^2}\varphi\left(\mathbf{r},\omega\right) = f\left(\mathbf{r},\omega\right)
\label{gf_eq:7}
\end{equation}
\renewcommand{\baselinestretch}{2} \small\normalsize

Equation \ref{gf_eq:7} is the Helmholtz equation and it represents the wave equation with only spatial derivatives. If we use the wavevector, $k_o = \omega/c$, and suppress the variable dependence for clarity, Equation \ref{gf_eq:7} can be rewritten as:

\begin{equation}
\nabla^2\varphi + k_o^2\varphi = f
\label{gf_eq:8}
\end{equation}
\renewcommand{\baselinestretch}{2} \small\normalsize

The Green's function for this problem can be found by replacing the driving function, $f$, with a delta function:

\begin{equation}
\nabla^2G+ k_o^2G = -\delta\left(\mathbf{r}-\mathbf{r}' \right)
\label{gf_eq:9}
\end{equation}
\renewcommand{\baselinestretch}{2} \small\normalsize

Equation \ref{gf_eq:9} shows that the Green's function is the solution to the differential equation for a spatial impulse and therefore represents the impulse response of the system. In electromagnetics, the driving force often has a negative sign as seen from the Laplace equation, $\nabla^2\varphi = -\rho/\epsilon_o$, so we typically chose a matching negative sign for the delta function. With this convention, we need to make sure the driving force has a negative sign when utilizing a Green's function later.

The general method to find Green's functions is to solve the homogeneous differential equation away from the delta function source and then integrate over a small region that includes the delta function for normalization. Some authors, such as \cite{jackson_classical_em}, include a $4\pi$ scale factor with the delta function to absorb this normalization term into the differential equation rather than the Green's function.

Equations \ref{gf_eq:8} and \ref{gf_eq:9} can be rearranged to solve for the Laplacian, so that $\nabla^2\varphi = f - k_o^2\varphi$ and $\nabla^2G = -\delta\left(\mathbf{r}-\mathbf{r}' \right) - k_o^2G$. We can substitute these values into Green's second identity (Equation \ref{gf_eq:5}), with $\psi=G$:

\begin{equation}
\begin{gathered}
\oint\limits_{S} \left[ \varphi\frac{\partial G}{\partial n} - G\frac{\partial \varphi}{\partial n} \right]dS = \int\limits_{V}d^3r' \left[ \varphi\nabla^2G- G \nabla^2 \varphi\right] \\
\oint\limits_{S} \left[ \varphi\frac{\partial G}{\partial n} - G\frac{\partial \varphi}{\partial n} \right]dS = \int\limits_{V}d^3r' \left[ \varphi \left(-\delta\left(\mathbf{r}-\mathbf{r}' \right) - k_o^2G\right)- G \left(f - k_o^2\varphi \right)\right] \\
\int\limits_{V}d^3r'\varphi\delta\left(\mathbf{r}-\mathbf{r}' \right) = \oint\limits_{S}\left[G\frac{\partial \varphi}{\partial n} - \varphi\frac{\partial G}{\partial n} \right]dS +\int\limits_{V}d^3r'\left[ Gk_o^2\varphi - \varphi k_o^2G-k_o^2G - Gf \right] \\
\int\limits_{V}d^3r'\varphi\delta\left(\mathbf{r}-\mathbf{r}' \right) = \oint\limits_{S}\left[G\frac{\partial \varphi}{\partial n} - \varphi\frac{\partial G}{\partial n} \right]dS -\int\limits_{V}d^3r' Gf
\end{gathered}
\label{gf_eq:10}
\end{equation}
\renewcommand{\baselinestretch}{2} \small\normalsize

As long as $\mathbf{r}'$ is in the volume $V$, we can use the sifting property of the delta function to provide an explicit representation for $\varphi$:

\begin{equation}
\boxed{\varphi = \oint\limits_{S}dS\left[G\frac{\partial \varphi}{\partial n} - \varphi\frac{\partial G}{\partial n} \right] -\int\limits_{V}d^3r' Gf}
\label{gf_eq:11}
\end{equation}
\renewcommand{\baselinestretch}{2} \small\normalsize

Equation \ref{gf_eq:11} defines the general solution for an inhomogeneous differential equation and the only assumption made here is that the forcing function will have a negative sign. The homogeneous solution is given by the surface integral and the response to a forcing function is given by the volume integral. When there are no boundary conditions (as in free space propagation), the normal derivatives are all zero so the surface integral vanishes and we are left with only the term due to the forcing function.

\begin{equation}
\varphi = -\int\limits_{V}d^3r' Gf
\label{gf_eq:11aa}
\end{equation}
\renewcommand{\baselinestretch}{2} \small\normalsize

In the simplest case, $f$ is a delta function with a scalar amplitude, $ f = -\alpha\delta(r'-r_o)$, so that

\begin{equation}
\varphi = \alpha G(r,r_o)
\label{gf_eq:11ab}
\end{equation}
\renewcommand{\baselinestretch}{2} \small\normalsize

\section {Boundary Conditions}
There are two traditional types of boundary conditions: Dirichlet, where $\varphi$ is specified on the boundary, or Neuman, where the normal derivative of $\varphi$ is specified on the boundary. There is a third type of boundary condition, known as the Robin condition, that is a weighted combination of Dirichlet and Neuman boundary conditions.

\subsection {Dirichlet Boundary Conditions}
For Dirichlet boundary conditions, the solution is specified on the surface, $\varphi\left(\mathbf{r}\right) |_{S} = \varphi\left(\mathbf{r}_S \right) = a\left(\mathbf{r}_s\right)$. To simplify Equation \ref{gf_eq:11}, we can enforce the following condition on the Green's function:

\begin{equation}
G\left(\mathbf{r},\mathbf{r}' \right)\bigg|_{S}=0
\label{gf_eq:12}
\end{equation}
\renewcommand{\baselinestretch}{2} \small\normalsize

Green's functions specified in this manner are sometimes called Green's functions of the first kind. Substituting Equation \ref{gf_eq:12} into Equation \ref{gf_eq:11} results in the following equation for $\varphi$ in the presence of Dirichlet boundary conditions.
\begin{equation}
\boxed{\varphi = -\oint\limits_{S}dS \varphi\frac{\partial G}{\partial n} -\int\limits_{V}d^3r' Gf}
\label{gf_eq:13}
\end{equation}
\renewcommand{\baselinestretch}{2} \small\normalsize

\subsection {Neuman Boundary Conditions}
For Neuman boundary conditions, the normal derivative of the field is specified on the surface, $\frac{\partial\varphi\left(\mathbf{r}\right)}{\partial n}|_{S} = b\left(\mathbf{r}_s\right)$. As described in \cite{jackson_classical_em}, \cite{zangwill_modern_em}, and \cite{balanis_advanced}, we cannot simply set the normal derivative of $G$ to $0$. This can be demonstrated by integrating Equation \ref{gf_eq:9} over a volume and then applying the divergence theorem:

\begin{equation}
\begin{gathered}
\int\limits_{V}d^3r' \nabla^2 G = \int\limits_{V}d^3r'\left[-\delta\left(\mathbf{r}-\mathbf{r}' \right) -k^2G \right] \\
\int\limits_{V}d^3r' \nabla \cdot \nabla G = -\int\limits_{V}d^3r'\delta\left(\mathbf{r}-\mathbf{r}' \right) -\int\limits_{V}d^3r'k^2G \\
\oint\limits_{S}\nabla G \cdot \hat{n} dS  = -1 - \int\limits_{V}d^3r'k^2G\\
\oint\limits_{S}\frac{\partial G}{\partial n} dS = -1 - \int\limits_{V}d^3r' k^2G
\end{gathered}
\label{gf_eq:14}
\end{equation}
\renewcommand{\baselinestretch}{2} \small\normalsize

Equation \ref{gf_eq:14} must hold for all values of $k$, so we can find a general condition for $\partial G/\partial n$ by letting $k = 0$:

\begin{equation}
\oint\limits_{S}\frac{\partial G}{\partial n} dS = -1
\label{gf_eq:15}
\end{equation}
\renewcommand{\baselinestretch}{2} \small\normalsize

Equation \ref{gf_eq:15} indicates that there is a discontinuity in the normal derivative of the Green's function. Letting $S$ be the total surface area gives us the following condition on the normal derivative of the Green's function:

\begin{equation}
\frac{\partial G}{\partial n} = -\frac{1}{S}
\label{gf_eq:16}
\end{equation}
\renewcommand{\baselinestretch}{2} \small\normalsize

Equation \ref{gf_eq:16} demonstrates the jump condition for the normal derivative of the Green's function. Note that when the surface is taken out to infinity that $\partial G/\partial n = 0$. Green's functions specified in this manner are sometimes called Green's functions of the second kind. Substituting Equation \ref{gf_eq:16} into Equation \ref{gf_eq:11} results in the following equation for $\varphi$ in the presence of Neuman boundary conditions.

\begin{equation}
\boxed{\varphi = \oint\limits_{S}dS\left[G\frac{\partial \varphi}{\partial n} + \frac{\varphi}{S} \right] -\int\limits_{V}d^3r' Gf}
\label{gf_eq:17}
\end{equation}
\renewcommand{\baselinestretch}{2} \small\normalsize

We can rewrite Equation \ref{gf_eq:17} as in \cite{jackson_classical_em}, using the average value of the solution over the surface, $\left< \varphi\right> = \frac{1}{S}\oint\limits_{S}\varphi dS$:

\begin{equation}
\boxed{\varphi = \left<\varphi \right> + \oint\limits_{S}G\frac{\partial \varphi}{\partial n}dS  -\int\limits_{V}d^3r' Gf}
\label{gf_eq:18}
\end{equation}
\renewcommand{\baselinestretch}{2} \small\normalsize

\subsection {Robin Boundary Conditions}
For Robin boundary conditions, a weighted linear combination of the solution and its derivative are specified on the surface, $\frac{\partial\varphi\left(\mathbf{r}\right)}{\partial n}|_{S} +\alpha\varphi\left(\mathbf{r}\right) |_{S}= c\left(\mathbf{r}_s\right)$. 

We can generally set $c=0$ so that the Robin boundary condition relates the Green's function on the surface to its normal derivative on the surface.

\begin{equation}
\frac{\partial G\left(\mathbf{r}\right)}{\partial n}\bigg|_{S} = -\alpha G
\label{gf_eq:18aa}
\end{equation}
\renewcommand{\baselinestretch}{2} \small\normalsize

Substituting Equation \ref{gf_eq:18aa} into Equation \ref{gf_eq:11} results in the following equation for $\varphi$ in the presence of Robin boundary conditions.

\begin{equation}
\boxed{\varphi = \oint\limits_{S}dS\left[\frac{\partial \varphi}{\partial n} + \alpha\varphi \right] -\int\limits_{V}d^3r' Gf}
\label{gf_eq:18aabb}
\end{equation}
\renewcommand{\baselinestretch}{2} \small\normalsize

Robin boundary conditions are commonly found in propagation problems with mixed polarization. They generally do not allow simple solutions are usually solved numerically.

\section {Properties of Green's Functions} \label{gf_sec:properties}
We have already discussed the fact that Green's functions represent the spatial impulse response of the differential equation. This section introduces some other properties of Green's functions, including causality, reciprocity, continuity, and general forms.

\subsection {Causality} \label{gf_sec:causality}
Green’s functions represent physical systems and therefore must be causal. With the time dependence as specified in Section \ref{gf_sec:time_dependence}, our solutions must have the following form:

\begin{equation}
G\left(r,t\right) \sim g\left(\omega t - k_or\right)
\label{gf_eq:18a}
\end{equation}
\renewcommand{\baselinestretch}{2} \small\normalsize

\subsection {Reciprocity} \label{gf_sec:reciprocity}
Green's functions must be symmetric with respect to $\mathbf{r}$ (the vector to the observer) and $\mathbf{r}'$ (the vector to the source), so that:

\begin{equation}
G\left(\mathbf{r},\mathbf{r}' \right) = G\left(\mathbf{r}',\mathbf{r} \right)
\label{gf_eq:18c}
\end{equation}
\renewcommand{\baselinestretch}{2} \small\normalsize

This property is known as reciprocity and it tells us that the impulse reponse for a propagating wave is the same in both directions. Reciprocity is commonly used in antenna radiation analysis as it allows us to analyze an antenna considering it as either a transmitter or a receiver.

\subsection {Continuity} \label{gf_sec:continuity}
A necessary condition for reciprocity is that Green's functions must be continuous in all space. However, they are singular so the derivative will have a discontinuity at $\mathbf{r} = \mathbf{r}'$.

\subsection {General Forms}
Because a Green's function is the spatial impulse response of the system, we can expect it to be used in a convolution integral. With that in mind, Green's functions usually take the form:
\begin{equation}
G\left(\mathbf{r},\mathbf{r}' \right) = g\left( \mathbf{r} - \mathbf{r}'\right)
\label{gf_eq:18d}
\end{equation}
\renewcommand{\baselinestretch}{2} \small\normalsize

Using the results of the previously described properties, the most general representation of a Green's function is as follows:

\begin{equation}
G\left(\mathbf{r},\mathbf{r}',t ,t'\right) = g\left(\omega |t-t'| - k| \mathbf{r} - \mathbf{r}' | \right)
\label{gf_eq:19b}
\end{equation}
\renewcommand{\baselinestretch}{2} \small\normalsize

\section {Free Space Green's Functions}
This section derives the free space Green's functions for the Helmholtz equation (Equation \ref{gf_eq:9}) in both 3-dimensions and 2-dimensions. In the 2-dimensional case, we will also derive the Green's function for the paraxial wave equation. In free space, there are no boundaries, so we can let $r\rightarrow \infty$ and use Dirichlet boundary conditions at $\infty$. Without loss of generality, we can place the delta function at the origin, so that $\delta\left(\mathbf{r}-\mathbf{r}' \right) \rightarrow \delta \left(\mathbf{r} \right)$. To find the Green's function, we need to solve the homogeneous equation away from the delta function and then integrate over a small region around the origin for normalization. The homogenous equation to solve is:

\begin{equation}
\nabla^2G+ k_o^2G = 0
\label{gf_eq:19}
\end{equation}
\renewcommand{\baselinestretch}{2} \small\normalsize

Because the delta function is rotationally symmetric, the Green's function must also be rotationally symmetric.

\subsection {3-Dimensions}\label{gf_sec:3d}
This section derives the free space Green's function for the Helmholtz equation in 3-dimensional space. Because we used the Fourier transform to remove the time dependence from the wave equation (Equation 
\ref{gf_eq:6}), we will first work in the frequency domain and then transform the result back to the time domain.

\subsubsection {Deriviation in the Frequency Domain}
In 3-dimensions,  rotational symmetry means that  $\partial G/\partial\theta = \partial G/\partial\phi=0$. Therefore we can expand the Laplacian of Equation \ref{gf_eq:19} in spherical coordinates and neglect the $\theta$ and $\phi$ components:

\begin{equation}
\frac{1}{r^2}\frac{\partial}{\partial r}\left(r^2\frac{\partial G}{\partial r}\right)+ k_o^2G = 0
\label{gf_eq:20}
\end{equation}
\renewcommand{\baselinestretch}{2} \small\normalsize

Expanding the derivatives in Equation \ref{gf_eq:20} along with a little algebraic manipulation allows us to rewrite this into something more manageable:

\begin{equation}
\begin{gathered}
\frac{1}{r^2}\left[2r\frac{\partial G}{\partial r}+ r^2\frac{\partial^2 G}{\partial r^2}  \right] + k_o^2G = 0 \\
r\frac{\partial^2 G}{\partial r^2} + 2\frac{\partial G}{\partial r}+ k_o^2rG = 0 \\
\frac{\partial^2 \left(rG\right)}{\partial r^2} + k_o^2rG = 0
\end{gathered}
\label{gf_eq:21}
\end{equation}
\renewcommand{\baselinestretch}{2} \small\normalsize

Equation \ref{gf_eq:21} has the following  solution:

\begin{equation}
G = c_1\frac{e^{jk_or}}{r} + c_2\frac{e^{-jk_or}}{r}
\label{gf_eq:22}
\end{equation}
\renewcommand{\baselinestretch}{2} \small\normalsize

From the causality restriction discussed in Section \ref{gf_sec:causality}, the $c_2$ term describes an outward propagating wave. This is the physical solution, which means $c_1$ must be equal to $0$. Therefore, the free space Green's function is:

\begin{equation}
G = c_2\frac{e^{-jk_or}}{r}
\label{gf_eq:23}
\end{equation}
\renewcommand{\baselinestretch}{2} \small\normalsize

This equation represents a spherical wave and is valid everywhere except $r=0$. To find $c_2$, we need to integrate Equation \ref{gf_eq:9} over a small volume centered at $r=0$ and then take the limit as $r\rightarrow0$.

\begin{equation}
\begin{gathered}
\lim_{r\to0}\int\limits_{V}d^3r'\left[\nabla^2G+ k_o^2G\right] = -\lim_{r\to0}\int\limits_{V}\delta\left(\mathbf{r}\right) \\
\lim_{r\to0}\int\limits_{V}d^3r'\left[\nabla \cdot\nabla G+ k_o^2G\right] = -1 \\
\lim_{r\to0}\left[\oint\limits_{S}\nabla G \cdot \hat{n} dS + \int\limits_{V}d^3r' k_o^2G\right] = -1 \\
\lim_{r\to0}\left[\oint\limits_{S}\frac{\partial G}{\partial n} dS + \int\limits_{V}d^3r' k_o^2G\right] = -1
\end{gathered}
\label{gf_eq:24}
\end{equation}
\renewcommand{\baselinestretch}{2} \small\normalsize

Substituting Equation \ref{gf_eq:23} in for $G$ and letting $\hat{n} = \hat{r}$, $dS = r'^2\sin{\theta'}d\theta' d\phi'$, and  $d^3r' = r'^2\sin{\theta'}dr'd\theta' d\phi'$ yields:

\begin{equation}
\begin{gathered}
\lim_{r\to0}\left[c_2\oint\limits_{S}\frac{\partial }{\partial r'}\frac{e^{-jk_or'}}{r'} r'^2\sin{\theta'} d\theta' d\phi '+ c_2\int\limits_{V}r'^2 \sin{\theta'}dr' d\theta' d\phi' k_o^2\frac{e^{-jk_or'}}{r'}\right] = -1 \\
\lim_{r\to0}\left[c_2 r^2\left( \frac{jk_o}{r} - \frac{1}{r^2}\right)e^{-jk_or}\oint\limits_{S}\sin{\theta'} d\theta' d\phi' + c_2\int\limits_{V}dr' d\theta' d\phi' k_o^2r'e^{-jk_or'}\sin{\theta'}\right] = -1 \\
\lim_{r\to0}\left[4\pi c_2 \left( jk_or - 1\right)e^{-jk_or}+ 4\pi c_2k_o^2\int\limits_{V}dr' r'e^{-jk_or'}\right] = -1 \\
\lim_{r\to0}4\pi c_2\left[\left( jk_or - 1\right)e^{-jk_or}+ k_o^2\int\limits_{V}dr' \frac{\partial}{\partial k_o}\frac{e^{-jk_or'}}{j}\right] = -1 \\
4\pi c_2\left[-1 +  \lim_{r\to0}\frac{k_o^2}{j}\frac{\partial}{\partial k_o}\left( \frac{-1}{jk_o}e^{-jk_or'}\right)\bigg|_0^{r}  \right] = -1 \\
-4\pi c_2\left[1 - \lim_{r\to0}k_o^2\frac{\partial}{\partial k_o}\left( \frac{1}{k_o}e^{-jk_o} - \frac{1}{k_o}\right) \right] = -1 \\
c_2\left[1 - \lim_{r\to0}k_o^2\left(-\frac{1}{k_o^2}+\frac{-jre^{-jk_or}}{k_o}+\frac{1}{k_o^2}\right) \right] = \frac{1}{4\pi} \\
 c_2 = \frac{1}{4\pi}
\end{gathered}
\label{gf_eq:25}
\end{equation}
\renewcommand{\baselinestretch}{2} \small\normalsize

\noindent Now letting $r \rightarrow |\mathbf{r}-\mathbf{r}'|$, we have the final free space 3-dimensional Green's function, $G_o$:

\begin{equation}
\boxed{G_o\left(\mathbf{r},\mathbf{r}'\right) = \frac{e^{-jk_o|\mathbf{r} - \mathbf{r}'|}}{4\pi |\mathbf{r} - \mathbf{r}'|}}
\label{gf_eq:26}
\end{equation}
\renewcommand{\baselinestretch}{2} \small\normalsize

As discussed for Equation \ref{gf_eq:9}, some authors, such as \cite{jackson_classical_em}, include the $4\pi$ scale factor with the delta function, so the free space 3-dimensional Green's function is:

\begin{equation}
G_o\left(\mathbf{r},\mathbf{r}'\right)  = \frac{e^{-jk_o|\mathbf{r} - \mathbf{r}'|}}{|\mathbf{r} - \mathbf{r}'|}
\label{gf_eq:27}
\end{equation}
\renewcommand{\baselinestretch}{2} \small\normalsize

\noindent In this case, the $4\pi$ scale factor would then appear in Equation \ref{gf_eq:11}.

To help visualize this Green's function, the magnitude and phase is shown in Figure \ref{gf_fig:1} and the real and imaginary components are shown in Figure \ref{gf_fig:2}. These figures constrain the Green's function to the $x-y$ plane at $z=0$ for visualization.

\begin{figure}[ht]
\begin{center}
\includegraphics[width=4in]{../media/3d_fs_gf_mag.png}
\end{center}
\renewcommand{\baselinestretch}{1}
\small\normalsize
\begin{quote}
\caption[Magnitude and Phase of 3-D Free Space Green's Function]{ Magnitude and Phase of 3-D Free Space Green's Function\label{gf_fig:1}}
\end{quote}
\end{figure} 
\renewcommand{\baselinestretch}{2}
\small\normalsize

\begin{figure}[ht]
\begin{center}
\includegraphics[width=4in]{../media/3d_fs_gf_re_im.png}
\end{center}
\renewcommand{\baselinestretch}{1}
\small\normalsize
\begin{quote}
\caption[Real and Imaginary Components of 3-D Free Space Green's Function]{Real and Imaginary Components of 3-D Free Space Green's Function \label{gf_fig:2}}
\end{quote}
\end{figure} 
\renewcommand{\baselinestretch}{2}
\small\normalsize

\subsubsection {Conversion to the Time Domain}
The Green's function in Equation \ref{gf_eq:26} is still in Fourier space. We can explicitly express the frequency dependence as:

\begin{equation}
G_o\left(\mathbf{r},\mathbf{r}',\omega\right) = \frac{e^{-j\frac{\omega}{c}|\mathbf{r} - \mathbf{r}'|}}{4\pi |\mathbf{r} - \mathbf{r}'|}
\label{gf_eq:28}
\end{equation}
\renewcommand{\baselinestretch}{2} \small\normalsize

\noindent To determine the Green's function in the time domain, we need to take the inverse Fourier transform.

\begin{equation}
\begin{gathered}
G_o\left(\mathbf{r},\mathbf{r}',t\right) = \frac{1}{2\pi}\int\limits_{-\infty}^{\infty}d\omega e^{j\omega t}G_o\left(\mathbf{r},\mathbf{r}',\omega\right) \\
G_o\left(\mathbf{r},\mathbf{r}',t\right) = \frac{1}{2\pi}\int\limits_{-\infty}^{\infty}d\omega e^{j\omega t}  \frac{e^{-j\frac{\omega}{c}|\mathbf{r}-\mathbf{r}'|}}{4\pi |\mathbf{r}-\mathbf{r}'|}\\
G_o\left(\mathbf{r},\mathbf{r}',t\right) = \frac{1}{4\pi |\mathbf{r}-\mathbf{r}'|}\frac{1}{2\pi}\int\limits_{-\infty}^{\infty}d\omega e^{-j\omega\left(\frac{|\mathbf{r}-\mathbf{r}'|}{c} - t\right)} \\
G_o\left(\mathbf{r},\mathbf{r}',t\right) = \frac{1}{4\pi |\mathbf{r}-\mathbf{r}'|}\frac{1}{2\pi}\int\limits_{-\infty}^{\infty}d\omega e^{-j\omega\left(t - \frac{|\mathbf{r}-\mathbf{r}'|}{c}\right)}
\end{gathered}
\label{gf_eq:29}
\end{equation}
\renewcommand{\baselinestretch}{2} \small\normalsize

\noindent Recognizing that the Fourier transform of the delta function is:

\begin{equation}
\delta(t-t_0) = \frac{1}{2\pi}\int\limits_{-\infty}^{\infty}d\omega e^{-j\omega \left(t-t_0\right)}
\label{gf_eq:30}
\end{equation}
\renewcommand{\baselinestretch}{2} \small\normalsize

\noindent We can write the free space Green's function in the temporal domain as:

\begin{equation}
\boxed{G_o\left(\mathbf{r},\mathbf{r}',t\right) = \frac{\delta\left(t-\frac{|\mathbf{r}-\mathbf{r}'|}{c} \right)}{4\pi |\mathbf{r}-\mathbf{r}'|}}
\label{gf_eq:31}
\end{equation}
\renewcommand{\baselinestretch}{2} \small\normalsize

Equation \ref{gf_eq:31} assumes the source is turned on at time $0$. If we let the source be turned on at an arbitrary time, $t'$, a more general expression is:

\begin{equation}
G_o\left(\mathbf{r},\mathbf{r}',t,t'\right) = \frac{\delta\left(|t-t'|-\frac{|\mathbf{r}-\mathbf{r}'|}{c} \right)}{4\pi |\mathbf{r}-\mathbf{r}'|}
\label{gf_eq:32}
\end{equation}
\renewcommand{\baselinestretch}{2} \small\normalsize

In the time domain, we generally refer to $G_o\left(\mathbf{r},\mathbf{r}',t, t'\right)$ as the retarded wave solution because an observer will experience the source as if it acted at an earlier (retarded) time, $t_1=|t-t'|-|\mathbf{r}-\mathbf{r}'|/c$.

\subsection {2-Dimensions}\label{gf_sec:2d}
This section derives the free space Green's function for the Helmholtz equation in 2-dimensional space. We will again work in the frequency domain first and then transform the result back to the time domain.

\subsubsection {Deriviation in the Frequency Domain}
In 2-dimensions,  rotational symmetry means that  $\partial G/\partial\phi =0$. Therefore we can expand the Laplacian of Equation \ref{gf_eq:19} in polar coordinates and neglect the $\phi$ components:

\begin{equation}
\frac{1}{\rho}\frac{\partial}{\partial \rho}\left(\rho\frac{\partial G}{\partial \rho}\right)+ k_o^2G = 0
\label{gf_eq:33}
\end{equation}
\renewcommand{\baselinestretch}{2} \small\normalsize

\noindent Equation \ref{gf_eq:33} is Bessel's equation with $\nu = 0$:

\begin{equation}
\frac{1}{\rho}\frac{\partial}{\partial \rho}\left(\rho\frac{\partial G}{\partial \rho}\right)+ \left(k_o^2 -\nu^2\right)G = 0
\label{gf_eq:34}
\end{equation}
\renewcommand{\baselinestretch}{2} \small\normalsize

\noindent The general solution to Equation \ref{gf_eq:33} is 

\begin{equation}
G = c_1J_0\left(k_o\rho\right) + c_2N_0\left(k_o\rho\right)
\label{gf_eq:35}
\end{equation}
\renewcommand{\baselinestretch}{2} \small\normalsize

Here $J_0$ is a Bessel function of the first kind, and $N_0$ is a Bessel function of the second kind. Since the delta function is singular at the origin, we cannot demand that $G$ be finite at the origin. This means that $c_2 \neq 0$ and we have one equation with two unknowns.

When working with propagating waves, it is sometimes more useful to work with Hankel functions than Bessel functions. The Hankel functions are:

\begin{equation}
\begin{gathered}
H_0^{(1)}(k\rho) = J_0(k_o\rho) + jN_0(k_o\rho) \\
H_0^{(2)}(k\rho) = J_0(k_o\rho) - jN_0(k_o\rho)
\label{gf_eq:36}
\end{gathered}
\end{equation}
\renewcommand{\baselinestretch}{2} \small\normalsize

The Hankel functions behave asymptotically like waves as can be seen by their behavior for large arguments \cite{abramowitz_stegun}, which can be derived from the integral representation through the saddle point method as shown in Appendix \ref{appendix_saddle_point_method}:

\begin{equation}
\begin{gathered}
H_0^{(1)}(k_o\rho) \approx \sqrt{\frac{2}{\pi k_o\rho}}e^{j\left(k_o\rho - \frac{\pi}{4}\right)}\\
H_0^{(2)}(k_o\rho) \approx \sqrt{\frac{2}{\pi k_o\rho}}e^{-j\left(k_o\rho - \frac{\pi}{4}\right)}
\label{gf_eq:36a}
\end{gathered}
\end{equation}
\renewcommand{\baselinestretch}{2} \small\normalsize

We can now represent the solution to Equation \ref{gf_eq:33} in terms of Hankel functions as:

\begin{equation}
G = c_1H_0^{(1)}\left(k_o\rho\right) +c_2H_0^{(2)}\left(k_o\rho\right) 
\label{gf_eq:37}
\end{equation}
\renewcommand{\baselinestretch}{2} \small\normalsize

From the causality discussion in Section \ref{gf_sec:causality}, $H_0^{(2)}$ acts like an outward propagating wave while $H_0^{(1)}$ acts like an inward propagating wave. Therefore, $H_0^{(2)}$ is the physical solution, $c_1=0$, and the Green's function is:

\begin{equation}
G = c_2H_0^{(2)}\left(k_o\rho\right) 
\label{gf_eq:38}
\end{equation}
\renewcommand{\baselinestretch}{2} \small\normalsize

As in Section \ref{gf_sec:3d}, we integrate Equation \ref{gf_eq:9} over a small disk, $D$, centered at $\rho = 0$ and take the limit as $\rho \rightarrow 0$. For small arguments, the asymptotic behavior of $H_0^{(2)}(k_o\rho) \sim -j\frac{2}{\pi}\ln\left({k_o\rho}\right)$ and we can substitute $G = -\frac{j2c_2}{\pi}\ln\left({k_o\rho}\right)$ \cite{abramowitz_stegun}. 

\begin{equation}
\begin{gathered}
\lim_{\rho\to 0}\int\limits_{D} \left[ \frac{1}{\rho'}\frac{\partial}{\partial \rho'}\left(\rho' \frac{\partial G}{\partial \rho'} \right) + k_o^2G\right]\rho' d\rho' d\theta' = \lim_{\rho\to 0}\int\limits_{D} \delta\left(\boldsymbol{\rho}-\boldsymbol{\rho}' \right)\rho' d\rho' d\phi' \\
\lim_{\rho\to 0}\int\limits_{D} \left[ -\frac{1}{\rho'}\frac{\partial}{\partial \rho'}\left(\rho' jc_1\frac{2}{\pi}\frac{\partial \ln(k_o\rho')}{\partial \rho'} \right) - k_o^2jc_1\frac{2}{\pi}\ln(k_o\rho')\right]\rho' d\rho' d\theta' = -1 \\
\lim_{\rho\to 0}-4jc_1\int\limits_{D} \left[\frac{\partial}{\partial \rho'}\left(\rho' \frac{\partial \ln(k_o\rho')}{\partial \rho'} \right) + k_o^2\rho'\ln(k_o\rho')\right] d\rho' = -1 \\
\lim_{\rho\to 0}4jc_1\left[\left(\rho' \frac{1 }{\rho'} \right)\bigg|_0^{\rho} + \int\limits_{D}k_o^2\rho'\ln(k_o\rho') d\rho'\right] = 1 \\
\lim_{\rho\to 0}c_1\left[ 1 +  k_o^2\left( \ln(k_o\rho')\frac{\rho'^2}{2}\bigg|_0^{\rho} - \int\limits_{D}\frac{\rho'^2}{2}\frac{1}{\rho'} d\rho' \right)\right] = -\frac{j}{4} \\
\lim_{\rho\to 0}c_1\left[ 1 +  k_o^2\left( \ln(k_o\rho)\frac{\rho^2}{2} - \frac{1}{2}\int\limits_{D}\rho' d\rho' \right)\right] = -\frac{j}{4} \\
\lim_{\rho\to 0}c_1\left[ 1 - \frac{\rho^2}{4}\right] = \frac{j}{4} \\
c_1 = -\frac{j}{4}
\end{gathered}
\label{gf_eq:39}
\end{equation}
\renewcommand{\baselinestretch}{2} \small\normalsize

Now letting $\rho \rightarrow |\boldsymbol{\rho}-\boldsymbol{\rho}'|$, we have the final free space 2-dimensional Green's function, $G_o$:

\begin{equation}
\boxed{G_o\left(\boldsymbol{\rho},\boldsymbol{\rho}'\right) = -\frac{j}{4}H_0^{(2)}\left(k_o|\boldsymbol{\rho} - \boldsymbol{\rho}' | \right)}
\label{gf_eq:40}
\end{equation}
\renewcommand{\baselinestretch}{2} \small\normalsize

To help visualize this Green's function, the magnitude and phase is shown in Figure \ref{gf_fig:3} and the real and imaginary components are shown in Figure \ref{gf_fig:4}. The real and imaginary components show the sinusoidal behavior of the Hankel functions for large arguments.

\begin{figure}[ht]
\begin{center}
\includegraphics[width=4in]{../media/2d_fs_gf_mag.png}
\end{center}
\renewcommand{\baselinestretch}{1}
\small\normalsize
\begin{quote}
\caption[Magnitude and Phase of 2-D Free Space Green's Function]{Magnitude and Phase of 2-D Free Space Green's Function \label{gf_fig:3}}
\end{quote}
\end{figure} 
\renewcommand{\baselinestretch}{2}
\small\normalsize

\begin{figure}[ht]
\begin{center}
\includegraphics[width=4in]{../media/2d_fs_gf_re_im.png}
\end{center}
\renewcommand{\baselinestretch}{1}
\small\normalsize
\begin{quote}
\caption[Real and Imaginary Components of 2-D Free Space Green's Function]{Real and Imaginary Components of 2-D Free Space Green's Function \label{gf_fig:4}}
\end{quote}
\end{figure} 
\renewcommand{\baselinestretch}{2}
\small\normalsize

\subsubsection {Conversion to the Time Domain}
As in Section \ref{gf_sec:3d}, the Green's function in Equation \ref{gf_eq:40} is still in Fourier space. We can explicitly express the frequency dependence as:

\begin{equation}
G_o\left(\boldsymbol{\rho},\boldsymbol{\rho}',\omega\right) = -\frac{j}{4}H_0^{(2)}\left(\frac{\omega}{c}|\boldsymbol{\rho} - \boldsymbol{\rho}' | \right)
\label{gf_eq:40a}
\end{equation}
\renewcommand{\baselinestretch}{2} \small\normalsize

To determine the Green's function in the time domain, we need to take the inverse Fourier transform:

\begin{equation}
\begin{gathered}
G_o\left(\boldsymbol{\rho},\boldsymbol{\rho}',t\right) = \frac{1}{2\pi}\int\limits_{-\infty}^{\infty}d\omega e^{j\omega t}G_o\left(\boldsymbol{\rho},\boldsymbol{\rho}',\omega\right) \\
G_o\left(\boldsymbol{\rho},\boldsymbol{\rho}',t\right) = -\frac{j}{4}\frac{1}{2\pi}\int\limits_{-\infty}^{\infty}d\omega e^{j\omega t} H_0^{(2)}\left(\frac{\omega}{c}|\boldsymbol{\rho} - \boldsymbol{\rho}' | \right)\\
\end{gathered}
\label{gf_eq:40b}
\end{equation}
\renewcommand{\baselinestretch}{2} \small\normalsize

We can use a representation of $H_0^{(2)}$ that is related to the Mehler-Sonine integral representation \cite{nist_handbook}:

\begin{equation}
H_o^{(2)}\left(z\right) = -\frac{1}{j\pi}\int\limits_{-\infty}^{\infty}e^{-jz\cosh(\tau)}d\tau = -\frac{2}{j\pi}\int\limits_{0}^{\infty}e^{-jz\cosh(\tau)}d\tau
\label{gf_eq:40c}
\end{equation}
\renewcommand{\baselinestretch}{2} \small\normalsize

\noindent Equation \ref{gf_eq:40c} is valid for $z>0$ . Since $k_o| \boldsymbol{\rho} - \boldsymbol{\rho}'| > 0$, this condition is satisfied. Letting $\rho' \rightarrow 0$ for simplification yields:

\begin{equation}
G_o\left(\boldsymbol{\rho},0,t\right) = \frac{1}{4\pi^2}\int\limits_{-\infty}^{\infty}d\omega e^{j\omega t} \int\limits_{0}^{\infty}e^{-j\frac{\omega}{c}\rho\cosh(\tau)}d\tau\\
\label{gf_eq:40d}
\end{equation}
\renewcommand{\baselinestretch}{2} \small\normalsize

\noindent We can bring the integral over $\omega$ inside the integral over $\tau$:

\begin{equation}
\begin{gathered}
G_o\left(\boldsymbol{\rho},0,t\right) = \frac{1}{4\pi^2}\int\limits_{0}^{\infty}d\tau\int\limits_{-\infty}^{\infty}d\omega e^{j\omega t} e^{-j\frac{\omega}{c}\rho\cosh(\tau)}\\
G_o\left(\boldsymbol{\rho},0,t\right) = \frac{1}{4\pi^2}\int\limits_{0}^{\infty}d\tau\int\limits_{-\infty}^{\infty}d\omega e^{-j\omega \left(\frac{\rho \cosh(\tau)}{c} - t\right)}\\
G_o\left(\boldsymbol{\rho},0,t\right) = \frac{1}{4\pi^2}\int\limits_{0}^{\infty}d\tau\int\limits_{-\infty}^{\infty}d\omega e^{-j\omega \left(t - \frac{\rho \cosh(\tau)}{c}\right)}\\
\end{gathered}
\label{gf_eq:40e}
\end{equation}
\renewcommand{\baselinestretch}{2} \small\normalsize

\noindent Again using the definition of the delta function from Equation \ref{gf_eq:30} yields:

\begin{equation}
G_o\left(\boldsymbol{\rho},0,t\right) = \frac{1}{2\pi}\int\limits_{0}^{\infty}d\tau \delta\left(t - \frac{\rho \cosh(\tau)}{c}\right)\\
\label{gf_eq:40f}
\end{equation}
\renewcommand{\baselinestretch}{2} \small\normalsize

For $G_o\left(\boldsymbol{\rho},0,t\right)$ to be nonzero, $t - \rho \cosh(\tau)/c > 0$ for all values of $\tau$. This means $t > \rho/c$ and we can enforce this condition on $t$ through the Heaviside step function, $H\left(t -\rho/c\right)$:

 \begin{equation}
G_o\left(\boldsymbol{\rho},0,t\right) = \frac{1}{2\pi}\int\limits_{0}^{\infty}d\tau H\left(t -\frac{\rho}{c}\right) \delta\left(t - \frac{\rho \cosh(\tau)}{c}\right)
\label{gf_eq:40g}
\end{equation}
 \renewcommand{\baselinestretch}{2} \small\normalsize
 
Because the argument of the delta function is a complicated function of $t$, we need to use the composition property of the delta function \cite{arfken_weber}, \cite{gbur_math}:

 \begin{equation}
\delta\left(g(x) \right) = \sum_{\substack{a \\g(a)=0}}\frac{\delta(x-a)}{|g'(a)|}
\label{gf_eq:40h}
\end{equation}
 \renewcommand{\baselinestretch}{2} \small\normalsize
 
From Equation \ref{gf_eq:40g}, $g(\tau) = t - \rho\cosh(\tau)/c$ and the only zero  is $\tau = \cosh^{-1}\left(ct/\rho\right)$. We can now rewrite the delta function from Equation \ref{gf_eq:40g} as:

 \begin{equation}
\delta\left(t - \frac{\rho \cosh(\tau)}{c}\right) = \frac{\delta\left(\tau -\cosh^{-1}\left(\frac{ct}{\rho} \right) \right)}{\rho\sinh\left(\cosh^{-1}\left(\frac{ct}{\rho} \right) \right)}
\label{gf_eq:40i}
\end{equation}
 \renewcommand{\baselinestretch}{2} \small\normalsize
 
\noindent We can now rewrite Equation \ref{gf_eq:40g} as:

 \begin{equation}
 \begin{gathered}
G_o\left(\boldsymbol{\rho},0,t\right) = \frac{1}{2\pi}\int\limits_{0}^{\infty}d\tau H\left(t -\frac{\rho}{c}\right)  \frac{c\delta\left(\tau -\cosh^{-1}\left(\frac{ct}{\rho} \right) \right)}{\rho\sinh\left(\cosh^{-1}\left(\frac{ct}{\rho} \right) \right)}\\
G_o\left(\boldsymbol{\rho},0,t\right) = \frac{cH\left(t -\frac{\rho}{c}\right)}{2\pi \rho\sinh\left(\cosh^{-1}\left(\frac{ct}{\rho} \right) \right)}\int\limits_{0}^{\infty}d\tau \delta\left(\tau -\cosh^{-1}\left(\frac{ct}{\rho} \right) \right)\\
G_o\left(\boldsymbol{\rho},0,t\right) = \frac{cH\left(t -\frac{\rho}{c}\right)}{2\pi \rho\sinh\left(\cosh^{-1}\left(\frac{ct}{\rho} \right) \right)}
\end{gathered}
\label{gf_eq:40j}
\end{equation}
 \renewcommand{\baselinestretch}{2} \small\normalsize
 
\noindent Using the identity that $\sinh\left(\cosh^{-1}(x) \right) = \sqrt{x^2 -1}$:

 \begin{equation}
 \begin{gathered}
G_o\left(\boldsymbol{\rho},0,t\right) = \frac{cH\left(t -\frac{\rho}{c}\right)}{2\pi \rho\sqrt{\left(\frac{ct}{\rho} \right)^2 - 1}     }\\
G_o\left(\boldsymbol{\rho},0,t\right) = \frac{H\left(t -\frac{\rho}{c}\right)}{2\pi \sqrt{t^2 - \left(\frac{\rho}{c}\right)^2}     }
\end{gathered}
\label{gf_eq:40k}
\end{equation}
 \renewcommand{\baselinestretch}{2} \small\normalsize
 
\noindent Letting $t\rightarrow |t-t'|$ and $\rho \rightarrow |\boldsymbol{\rho} - \boldsymbol{\rho}'|$ yields the final result:

 \begin{equation}
\boxed{G_o\left(\boldsymbol{\rho},\boldsymbol{\rho}',t,t'\right) = \frac{H\left(|t-t'| -\frac{|\boldsymbol{\rho} - \boldsymbol{\rho}'|}{c}\right)}{2\pi \sqrt{|t-t'|^2 -\left(\frac{|\boldsymbol{\rho} - \boldsymbol{\rho}'|}{c}\right)^2 }     }}
\label{gf_eq:40l}
\end{equation}
\renewcommand{\baselinestretch}{2} \small\normalsize

\subsection {Paraxial Wave Equation} \label{gf_sec:paraxial}
In this section we derive the Green's function for the paraxial wave equation in the frequency domain.

\subsubsection {Derivation of the Paraxial Wave Equation}
We can derive the paraxial from the Helmholtz equation by assuming a wave propagating in the $x$ direction of the form $U(x,z) = A(x,z)e^{-jk_ox}$.

 \begin{equation}
 \begin{gathered}
 \left[ \nabla^2 + k_o^2\right]U = 0 \\
\left[\frac{\partial^2 }{\partial x^2} + \frac{\partial^2 }{\partial z^2} + k_o^2\right]U = 0 \\
\frac{\partial }{\partial x}\left(\frac{\partial U}{\partial x} \right) + \frac{\partial^2 U}{\partial z^2} + k_o^2 U = 0 \\
\frac{\partial }{\partial x}\left(-jk_oAe^{-jk_ox}+e^{-jk_ox}\frac{\partial A}{\partial x} \right) + e^{-jk_ox}\frac{\partial^2 A}{\partial z^2} + k_o^2 Ae^{-jk_ox} = 0 \\
-k_o^2Ae^{-jk_ox} -2jk_oe^{-jk_ox}\frac{\partial A}{\partial x}+e^{-jk_ox}\frac{\partial^2 A}{\partial x^2} + e^{-jk_ox}\frac{\partial^2 A}{\partial z^2} + k_o^2 Ae^{-jk_ox} = 0 \\
e^{-jk_ox}\left( -2jk_o\frac{\partial A}{\partial x}+\frac{\partial^2 A}{\partial x^2} + \frac{\partial^2 A}{\partial z^2}\right) = 0 \\
\end{gathered}
\label{gf_eq:41}
\end{equation}
 \renewcommand{\baselinestretch}{2} \small\normalsize
 
\noindent This gives us an equation dependent only on the amplitude, $A$

 \begin{equation}
-2jk_o\frac{\partial A}{\partial x}+\frac{\partial^2 A}{\partial x^2} + \frac{\partial^2 A}{\partial z^2} = 0 \\
\label{gf_eq:42}
\end{equation}
 \renewcommand{\baselinestretch}{2} \small\normalsize
 
 We can now apply the paraxial approximation which states that the change in the modulation function along the propagation direction is negligible over a wavelength. This can be expressed mathematically as
 
  \begin{equation}
k_o\frac{\partial A}{\partial x} >> \frac{\partial^2 A}{\partial^2 x}
\label{gf_eq:43}
\end{equation}
 \renewcommand{\baselinestretch}{2} \small\normalsize
 
 With this approximation we can neglect the 2nd derivative in $x$ and write the paraxial equation as
 
\begin{equation}
\boxed{-2jk_o\frac{\partial A}{\partial x} + \frac{\partial^2 A}{\partial z^2} = 0} \\
\label{gf_eq:44}
\end{equation}
\renewcommand{\baselinestretch}{2} \small\normalsize
 
The paraxial wave equation is a parabolic differential equation with the same structure as the diffusion equation. However, since the effective diffusion coefficient is complex, we can expect the solution to oscillate rather than decay exponentially.

Since we have reduced the original Helmholtz equation for the full propagating wave to one that is only dependent on the modulation function, we will need to scale the Green's function by $e^{-jk_ox}$ to get the full propagating wave solution.
 
\subsubsection {Green's Function Derivation in the Frequency Domain}
As stated in the previous section, the Green's function for the paraxial wave equation, $G_p$, will only define the amplitude modulation. To get the full Green's function, we will need to scale by the plane wave, $G_o= G_p e^{-jk_ox}$. We can start with substituting the Greens function into equation \ref{gf_eq:44}.

\begin{equation}
-2jk_o\frac{\partial G_p}{\partial x} + \frac{\partial^2 G_p}{\partial z^2} = -\delta(x)\delta(z)
\label{gf_eq:44a}
\end{equation}
 \renewcommand{\baselinestretch}{2} \small\normalsize
 
We can then take the Fourier transform of Equation \ref{gf_eq:44a} along $z$ and multiply both sides by $-1$.

\begin{equation}
2jk_o\frac{\partial \hat{G_p}}{\partial x} +k^2\hat{G_p} = \delta(x)
\label{gf_eq:45}
\end{equation}
 \renewcommand{\baselinestretch}{2} \small\normalsize
 
Solving away from the delta function yields the following expression for $\hat{G_p}$:

\begin{equation}
\hat{G_p}= ce^{\frac{jk^2}{2k_o}x}
\label{gf_eq:46}
\end{equation}
 \renewcommand{\baselinestretch}{2} \small\normalsize
 
To find $c$, we can integrate Equation \ref{gf_eq:45} over a small region $\pm\epsilon$ and take the limit as $x\rightarrow 0$.

\begin{equation}
\lim_{x\rightarrow 0}\int_{-\epsilon}^{\epsilon}dx\left[2jk_o\frac{\partial\hat{G_p}}{\partial x}+ k^2\hat{G_p} \right] = \lim_{x\rightarrow 0}\int_{-\epsilon}^{\epsilon}dx\delta(x)\\
\label{gf_eq:47}
\end{equation}
 \renewcommand{\baselinestretch}{2} \small\normalsize
 
\noindent $\hat{G_p}$ must be continuous, so the integral over the $k^2\hat{G_p}$ term must be 0.

\begin{equation}
\begin{gathered}
\lim_{x\rightarrow 0}2jk_o\int_{-\epsilon}^{\epsilon}dx\frac{\partial\hat{G_p}}{\partial x}= 1\\
\lim_{x\rightarrow 0}2jk_o\hat{G_p} = 1 \\
\lim_{x\rightarrow 0}2jk_oce^{\frac{jk^2}{2k_o}x} = 1\\
c = \frac{1}{j2k_o}\\
\end{gathered}
\label{gf_eq:11cb}
\end{equation}
 \renewcommand{\baselinestretch}{2} \small\normalsize
 
\noindent This yields the final expression for $\hat{G}$:

\begin{equation}
\boxed{\hat{G_p}= \frac{1}{2jk_o}e^{\frac{jk^2}{2k_o}x}}
\label{gf_eq:11cc}
\end{equation}
 \renewcommand{\baselinestretch}{2} \small\normalsize
 
\noindent To find $G_p$, we can take the inverse Fourier transform of $\hat{G_p}$:

\begin{equation}
\begin{aligned}
G_p &= \mathcal{F}^{-1}\{\hat{G_p}\} = \frac{1}{2jk_o}\frac{1}{2\pi}\int_{-\infty}^{\infty}dk e^{\frac{jk^2}{2k_o}x}e^{-jkz} \\
& = \frac{1}{2jk_o}\frac{1}{2\pi}\int_{-\infty}^{\infty}dk e^{\frac{jk^2}{2k_o}x-jkz} \\
\end{aligned}
\label{gf_eq:11d}
\end{equation}
 \renewcommand{\baselinestretch}{2} \small\normalsize
 
\noindent To solve this, we need to complete the square with respect to $k$

\begin{equation}
\begin{gathered}
\frac{jx}{2k_o}\left[k^2  -2k\frac{k_oz}{x}\right]\\
\frac{jx}{2k_o}\left[\left(k - \frac{k_oz}{x}\right)^2 - \frac{k_o^2z^2}{x^2} \right]\\
\frac{jx}{2k_o}\left(k - \frac{k_oz}{x}\right)^2 - \frac{jk_oz^2}{x}\\
\end{gathered}
\label{gf_eq:11e}
\end{equation}
 \renewcommand{\baselinestretch}{2} \small\normalsize
 
\noindent Now we can substitute into Equation \ref{gf_eq:11d}

\begin{equation}
\begin{aligned}
G_p &= \frac{1}{2jk_o}\frac{1}{2\pi}\int_{-\infty}^{\infty}dk e^{\frac{jx}{2k_o}\left(k  -\frac{k_oz}{x}\right)^2- \frac{jk_o}{2x}z^2 } \\
&= \frac{1}{2jk_o}\frac{1}{2\pi} \sqrt{\frac{\pi j2k_o}{x}}e^{-j\frac{k_o}{2x}z^2 } \\
&= \frac{1}{2jk_o}\sqrt{\frac{jk_o}{2\pi x}}\exp\left[-j\frac{k_oz^2}{2x} \right]\\
&= \sqrt{\frac{1}{8\pi j k_ox}}\exp\left[-j\frac{k_oz^2}{2x} \right]\\
\end{aligned}
\label{gf_eq:11f}
\end{equation}
 \renewcommand{\baselinestretch}{2} \small\normalsize
 
\noindent Multiplying by $e^{-jk_ox}$ yields the full Green's function.

\begin{equation}
G= \sqrt{\frac{1}{8\pi jk_ox}}\exp\left[-j\frac{k_oz^2}{2x} \right]e^{-jk_ox}
\label{gf_eq:11fa}
\end{equation}
  \renewcommand{\baselinestretch}{2} \small\normalsize
  
We can let $x\rightarrow |x-x'|$ and $x\rightarrow |z-z'|$ to represent the Green's function as traditionally shown.

\begin{equation}
\boxed{G\left(x,x',z,z' \right)= \sqrt{\frac{1}{8\pi jk_o|x-x'|}}\exp\left[-jk_o\left(|x-x'| + \frac{|z-z'|^2}{2|x-x'|}\right) \right]}\\
\label{gf_eq:11fb}
\end{equation}
 \renewcommand{\baselinestretch}{2} \small\normalsize
 
To help visualize this Green's function, the magnitude and phase is shown in Figure \ref{gf_fig:5} and the real and imaginary components are shown in Figure \ref{gf_fig:6}. Both figures show a slice of the Green's function along $z = 0$. 

\begin{figure}[ht]
\begin{center}
\includegraphics[width=4in]{../media/2d_paraxial_gf_mag.png}
\end{center}
\renewcommand{\baselinestretch}{1}
\small\normalsize
\begin{quote}
\caption[Magnitude and Phase of 2-D Paraxial Green's Function]{Magnitude and Phase of 2-D Paraxial Green's Function \label{gf_fig:5}}
\end{quote}
\end{figure} 
\renewcommand{\baselinestretch}{2}
\small\normalsize

\begin{figure}[ht]
\begin{center}
\includegraphics[width=4in]{../media/2d_paraxial_gf_re_im.png}
\end{center}
\renewcommand{\baselinestretch}{1}
\small\normalsize
\begin{quote}
\caption[Real and Imaginary Components of 2-D Paraxial Green's Function]{Real and Imaginary Components of 2-D Paraxial Green's Function \label{gf_fig:6}}
\end{quote}
\end{figure} 
\renewcommand{\baselinestretch}{2}
\small\normalsize

\subsubsection{Derivation From 3-D Free Space Green's Function}
A simpler method to derive the Green's function for the paraxial wave equation is to apply the paraxial assumption to the 3-D free space Green's function. This assumption allows us to substitute $r$ with $x$ for the amplitude, but we will need to integrate out the $y$ dependence of the phase because it is a rapidly oscillating function. We can use a binomial expansion to separate the $y$ and $z$ dependence.

\begin{equation}
r = \sqrt{x^2+y^2+z^2} \approx x + \frac{y^2}{2x}+\frac{z^2}{2x}
\label{gf_eq:11za}
\end{equation}
 \renewcommand{\baselinestretch}{2} \small\normalsize
 
\noindent The paraxial approximation of the 3-D free space Green's function is then

\begin{equation}
G_o \approx \frac{\exp\left[-jk_o\left(x+\frac{z^2}{2x} \right)\right]}{4\pi x} \exp\left[-jk_o\frac{y^2}{2x}\right]
\label{gf_eq:11zb}
\end{equation}
 \renewcommand{\baselinestretch}{2} \small\normalsize
 
\noindent We can integrate out the $y$ dependence

\begin{equation}
\int_{-\infty}^{\infty} dy \exp\left[-jk_o\frac{y^2}{2x}\right] = \sqrt{\frac{2\pi x}{jk_o}}
\label{gf_eq:11zc}
\end{equation}
 \renewcommand{\baselinestretch}{2} \small\normalsize
 
\noindent This then scales Equation \ref{gf_eq:11zb}

\begin{equation}
\begin{aligned}
G_o &= \frac{\exp\left[-jk_o\left(x+\frac{z^2}{2x} \right)\right]}{4\pi x} \sqrt{\frac{2\pi x}{jk_o}}\\
&= \sqrt{\frac{1}{j8\pi k_o x}}\exp\left[-jk_o\left(x+\frac{z^2}{2x} \right)\right]
\label{gf_eq:11zd}
\end{aligned}
\end{equation}
 \renewcommand{\baselinestretch}{2} \small\normalsize
 
\noindent Equation \ref{gf_eq:11zb} is identical to Equation \ref{gf_eq:11fa} so we get the same answer either way.

\subsection {Tabularized Green's Functions}
Table \ref{gf_tab:0} collects the various Green's functions derived in this section. Again, these were derived assuming a time dependence going as $e^{j\omega t}$.
\begin{table}[ht]
  \begin{center}
      \renewcommand{\baselinestretch}{1} \small\normalsize
  \begin{quote}
    \caption[Table of Derived Green's Functions]{Table of Derived Green's Functions\label{gf_tab:0}}
  \end{quote}
  \begin{tabular} {|c | c | c|}
    \hline
  \bf{Wave Equation} & \bf{Frequency Domain} & \bf{Time Domain}\\ \hline
  3-D Free Space & $\displaystyle\frac{e^{-jk_o|\mathbf{r} - \mathbf{r}'|}}{4\pi |\mathbf{r} - \mathbf{r}'|}$ &  $\displaystyle\frac{\delta\left(|t-t'|-\frac{|\mathbf{r}-\mathbf{r}'|}{c} \right)}{4\pi |\mathbf{r}-\mathbf{r}'|}$ \\ \hline
  2-D Free Space & $\displaystyle -\frac{j}{4}H_0^{(2)}\left(k_o|\boldsymbol{\rho} - \boldsymbol{\rho}' | \right)$ & $\displaystyle\frac{H\left(|t-t'| -\frac{|\boldsymbol{\rho} - \boldsymbol{\rho}'|}{c}\right)}{2\pi \sqrt{|t-t'|^2 -\left(\frac{|\boldsymbol{\rho} - \boldsymbol{\rho}'|}{c}\right)^2 }     }$  \\ \hline
  2-D Paraxial & $\displaystyle\sqrt{\frac{1}{8\pi jk_o|x-x'|}}\exp\left[-jk_o\left(|x-x'| + \frac{|z-z'|^2}{2|x-x'|}\right) \right]$ & N/A \\ \hline
\end{tabular}
\end{center}
\end{table}
\renewcommand{\baselinestretch}{2} \small\normalsize

\section {Propagation With Green's Functions}
This section describes how to use Green's functions to find solutions for propagation problems. We will identify the disturbance at the observation point as $U(P_2)$ and the disturbance at the source as $U(P_1)$.

\subsection {Free Space Propagation}
For free space propagation problems, there are no boundaries present and we only have the forcing function of Equation \ref{gf_eq:11} surviving so that $U(P_0) = -\int d^3r' Gf$. 

\subsubsection{Point Source}
The simplest implementation is to assume a point source, so that $f= -\alpha \delta(\mathbf{r}-\mathbf{r_o})\delta(t-t_o)$, in which case the solution simply becomes $U = G(\mathbf{r},\mathbf{r}_o,t,t_o)$. In the 3-dimensional case, the disturbance is then $U(P_0) =1/(4\pi r)\exp[-jk_or]$.

The point source approximation greatly simplifies the calculation and is adequate to capture propagation characteristics in many cases, especially at long ranges. At closer ranges, the distributed aspects of a source become apparent and may be important to include in cases such as predicting the infrared signature of objects.

\subsubsection {Diffraction through an Aperture}
The canonical diffraction problem is the propagation of light through an aperture and is heavily utilized in the field of Fourier Optics \cite{goodman_fourier} \cite{gaskill_fourier}.

\subsection{Rayleigh-Sommerfeld Diffraction Integral}
Following Huygen's principle, we can use every point in the aperture as a source for secondary waves. In order to ensure the integral over the surface vanishes at far distances, we need to apply the Sommerfeld radiation condition on the disturbance $U$

\begin{equation}
 \lim_{R\to\infty} R\left(\frac{\partial U}{\partial n} -jkU \right) = 0.
\label{gf_eq:48}
\end{equation}
\renewcommand{\baselinestretch}{2} \small\normalsize

We will assume there is no additional forcing function in the aperture so the solution from Equation \ref{gf_eq:11} is just the surface integral over the aperture.

The Rayleigh-Sommerfeld approach uses an alternative Green's function that is generated by a pair of mirrored point sources located at $\mathbf{r}_1$ and $\mathbf{r}_2$ such that $\mathbf{r}_1 = -\mathbf{r}_2$. This configuration ensures that the Green's function vanishes on the surface so the boundary conditions do not need to be applied to both $U$ and $\partial U/\partial n$. This Greens function, $G\_$, is then given as

\begin{equation}
G\_= \frac{\exp[-jk_or_1]}{4\pi r_1} - \frac{\exp[-jk_or_2]}{4\pi r_2}
\label{gf_eq:49}
\end{equation}
\renewcommand{\baselinestretch}{2} \small\normalsize

With $\theta_1$ equal to the angle between $\hat{n}$ and $\mathbf{r_1}$ and $\theta_2$ equal to the angle between $\hat{n}$ and $\mathbf{r_2}$, we can write the normal derivative, $\partial G\_/\partial n$, as 

\begin{equation}
\frac{\partial G\_}{\partial n}
=\cos(\theta_1)\left(-jk_o - \frac{1}{r_1} \right)\frac{\exp[-jk_or_1]}{4\pi r_1} -\cos(\theta_2)\left(-jk_o - \frac{1}{r_2} \right)\frac{\exp[-jk_or_1]}{4\pi r_2}
\label{gf_eq:50}
\end{equation}
\renewcommand{\baselinestretch}{2} \small\normalsize

\noindent In the aperture, $\cos(\theta_1) = -\cos(\theta_2)$ and $r_1=r_2=r$ so that

\begin{equation}
\begin{aligned}
\frac{\partial G\_}{\partial n}\bigg|_S &= -2\cos(\theta)\left(\frac{jk_o}{4\pi r} + \frac{1}{4\pi r^2}\right)\exp[-jk_or]\\
&=2\frac{\partial G}{\partial n}\bigg|_S \\
\end{aligned}
\label{gf_eq:51}
\end{equation}
\renewcommand{\baselinestretch}{2} \small\normalsize

\noindent With this Green's function in hand, we can rewrite the solution in Equation \ref{gf_eq:11} as

\begin{equation}
\begin{gathered}
U(P_2) = -\oint\limits_{S}U(S)\frac{\partial G\_}{\partial n}dS\\
= -2\oint\limits_{S}U(S)\frac{\partial G}{\partial n}dS
\end{gathered}
\label{gf_eq:52}
\end{equation}
\renewcommand{\baselinestretch}{2} \small\normalsize

This equation directly relates the disturbance at $P_2$ to the disturbance in the aperture. For the 3-dimensional case with the assumption that $1/r^2 << k_o$, we get the following expression for the Rayleigh-Sommerfeld formula, which is the mathematical representation of the Huygens-Fresnel principle.

\begin{equation}
\boxed{U(P_2) =\frac{j}{\lambda}\int_S U(S)\frac{\exp\left[-jk_o r\right]}{r}\cos(\theta)dS}
\label{gf_eq:53}
\end{equation}
\renewcommand{\baselinestretch}{2} \small\normalsize

As stated in \cite{goodman_fourier} and \cite{gaskill_fourier}, the Huygens-Fresnel principle states that the disturbance observed at a given point is the superposition of diverging spherical waves that originate at secondary points in the aperture. The diverging spherical waves are reduced in amplitude by a factor $\lambda$, scaled by a directivity pattern $\cos(\theta)$, and shifted in phase by $90^{\circ}$.

\noindent If we let $r \rightarrow |\mathbf{r}-\mathbf{r}'|$ for generality, the convolution property becomes more apparent

\begin{equation}
U(P_2) =\frac{j}{\lambda}\int_S U(S)\frac{\exp\left[-jk_o |\mathbf{r}-\mathbf{r}'|\right]}{|\mathbf{r}-\mathbf{r}'|}\cos(\theta)dS
\label{gf_eq:53a}
\end{equation}
\renewcommand{\baselinestretch}{2} \small\normalsize

Equation \ref{gf_eq:53a} tells us that propagation acts as a linear system with the convolution kernel, $h$, given as

\begin{equation}
h=\frac{j}{\lambda}\frac{\exp\left[-jk_o |\mathbf{r}-\mathbf{r}'|\right]}{|\mathbf{r}-\mathbf{r}'|}\cos(\theta)
\label{gf_eq:53b}
\end{equation}
\renewcommand{\baselinestretch}{2} \small\normalsize

\subsubsection{Fresnel Diffraction}
The Fresnel approximation is essentially the paraxial approximation but retaining all 3 dimensions. If we take a binomial expansion of $r$ but keep both the $x$ and $y$ components and assume the source plane is located at $z=0$, we can approximate $r$ as

\begin{equation}
r\approx z + \frac{(x-x’)^2}{2z}+\frac{(y-y’)^2}{2z}
\label{gf_eq:53c}
\end{equation}
\renewcommand{\baselinestretch}{2} \small\normalsize

Here, $x'$ and $y'$ represent the coordinates in the source plane and $x$ and $y$ represent the coordinates in the observation plane. With the substitution $\cos(\theta) = z/r$, we can now rewrite Equation \ref{gf_eq:53} as

\begin{equation}
\boxed{U(P_2) =\frac{je^{-jk_oz}}{\lambda z}\int U(x’,y’)\exp\left[-j \frac{k}{2z}\left([x-x’]^2 + [y-y’]^2 \right) \right]dx’ dy’}
\label{gf_eq:53d}
\end{equation}
\renewcommand{\baselinestretch}{2} \small\normalsize

Equation \ref{gf_eq:53d} is the Fresnel approximation of the diffraction integral and represents propagation as convolution with a kernel containing a quadratic phase factor

\begin{equation}
h = \frac{je^{-jk_o z}}{\lambda z}\exp\left[-j\frac{k_o}{2z}\left(x^2 + y^2 \right) \right]
\label{gf_eq:53e}
\end{equation}
\renewcommand{\baselinestretch}{2} \small\normalsize

From \cite{goodman_fourier}, the Fresnel approximation is usually sufficient when $z\geq D_{\text{max}}^2/16$, where $D_{\text{max}}$ is the maximum spatial extent, $D_{\text{max}} = x_{\text{max}}^2 + y_{\text{max}}^2$. 

\subsubsection{Fraunhofer Diffraction}
If we allow the wave to propagate further, the quadratic phase factor will not contribute significantly and we can factor it out of the integral. This approximation is valid when the propagation distance meets the following criteria.

\begin{equation}
z >> \frac{k_oD_{\text{max}}}{2}
\label{gf_eq:53f}
\end{equation}
\renewcommand{\baselinestretch}{2} \small\normalsize

For clarity, we typically replace $x'$ with $\xi$ and $y'$ with $\eta$ so that Equation \ref{gf_eq:53d} becomes

\begin{equation}
U(P_2) =\frac{je^{-jk_oz}}{\lambda z}\int U(\xi,\eta)\exp\left[-j \frac{k}{2z}\left([x-\xi]^2 + [y-\eta]^2 \right) \right]d\xi d\eta
\label{gf_eq:53g}
\end{equation}
\renewcommand{\baselinestretch}{2} \small\normalsize

We can neglect the $\xi^2$ and $\eta^2$ terms in the exponential and let $(x-\xi)^2 \approx x^2-2x\xi$ and $(y-\eta)^2\approx y^2-2y\eta$. We can factor out the quadratic phase in terms of observation coordinates, $x$ and $y$, and rewrite Equation \ref{gf_eq:53g} as

\begin{equation}
\boxed{U(P_2) =\frac{je^{-jk_oz}e^{-j\frac{k_o}{2z}(x^2+y^2)}}{\lambda z}\int U(\xi,\eta)\exp\left[j2\pi\left(\frac{x}{\lambda z}\xi + \frac{y}{\lambda z}\eta \right) \right]d\xi d\eta}
\label{gf_eq:53h}
\end{equation}
\renewcommand{\baselinestretch}{2} \small\normalsize

Equation \ref{gf_eq:53h} is the Fraunhofer approximation and is simply the spatial Fourier transform of the aperture with spatial frequencies $f_x = x/(\lambda z)$ and $f_y = y/(\lambda z)$ \cite{goodman_fourier} \cite{gaskill_fourier}. In other words, the spatial frequencies are the source coordinates at the observation plane divided by the product of the wavelength with the propagation distance.

\subsection {Diffraction from a Planar Surface}
We can treat diffraction from a planar surface in the same fashion as diffraction through an aperture but we will need to include a reflection coefficient, $\Gamma$.

\subsubsection {Rough Sea Surface}
In maritime environments, we need to propagate over the ocean, which can be approximated as a planar surface. We will derive the propagation integral for a wave that is reflected off an ocean surface with small random components using a paraxial Green's function. The geometry is shown in Figure \ref{gf_fig:15}, where $\tilde{x}$, is the region over which the energy is significantly reflected, $x_m$ is the primary point of reflection, $L_1$ is the direct path, $L_2$ and $L_3$ are the path lengths for the various reflected rays, $s(x)$ is the altitude of the sea surface at $x$, $h_2$ is the altitude of the receiver at $P_2$, $h_1$ is the altitude of the transmitter at $P_1$, and $L$ is the ground distance between $P_1$ and $P_2$. 

\begin{figure}[ht]
  \begin{center}
\includegraphics[width=5in]{../media/analysis/multipath_layout.png}
  \end{center}
  \renewcommand{\baselinestretch}{1} \small\normalsize
  \begin{quote}
    \caption[Geometry for Diffraction Along the Sea Surface]{Geometry for Diffraction Along the Sea Surface\label{gf_fig:15}}
  \end{quote}
\end{figure}
\renewcommand{\baselinestretch}{2} \small\normalsize

We can use Equation \ref{gf_eq:52} with the paraxial Green's function. The surface normal is now given as $\hat{z}$, so we only need to take the derivative with respect to $z$. The reflected wave along $L_3$ will then be given by

\begin{equation}
\begin{aligned}
U(P_2) &= 2jk_o\int\limits_{0}^{L}dx\Gamma U(s)\frac{z}{x}\sqrt{\frac{1}{8\pi jk_o x}}\exp\left[-jk_o\left(x +\frac{z^2}{2x} \right) \right] \\
&= \int\limits_{0}^{L}dx\Gamma U(s)\frac{z}{x}\sqrt{\frac{jk_o}{2\pi x}}\exp\left[-jk_o\left(x +\frac{z^2}{2x} \right) \right] \\
\end{aligned}
\label{gf_eq:55}
\end{equation}
\renewcommand{\baselinestretch}{2} \small\normalsize

From the geometry in Figure \ref{gf_fig:15}, we are shifting the starting point for the return path by $x$ and we need to let $x \rightarrow L-x$ and $z \rightarrow h_2-s$ so that we now have

\begin{equation}
\begin{aligned}
U_2(P_2) &= \int\limits_{0}^{L}dx\Gamma U(s)\frac{h_2-s}{L-x}\sqrt{\frac{jk_o}{2\pi (L-x)}}\exp\left[-jk_o\left(L-x +\frac{(h_2-s)^2}{2(L-x)} \right) \right] \\
&= \int\limits_{0}^{L}dx\Gamma U(s)\frac{h_2-s}{L-x}\sqrt{\frac{jk_o}{2\pi (L-x)}}\exp\left[-jk_oL_3\right] \\
\end{aligned}
\label{gf_eq:56}
\end{equation}
\renewcommand{\baselinestretch}{2} \small\normalsize

The reflected wave solution at the sea surface, $U_2(s)$, follows from the free space solution, where $z \rightarrow h_1-s$.

\begin{equation}
\begin{aligned}
U_2(s) &= \sqrt{\frac{1}{8\pi jk_ox}}\exp\left[-jk_oL_2\right]\\
\end{aligned}
\label{gf_eq:57}
\end{equation}
\renewcommand{\baselinestretch}{2} \small\normalsize

\noindent The full solution for the reflected wave is then 
\begin{equation}
\begin{aligned}
U_2(P_2) &= \int_0^L dx \Gamma \sqrt{\frac{1}{8\pi j k_o x}}\exp[-jk_oL_2]\frac{h_2-s}{L-x}\sqrt{\frac{jk_o}{2\pi(L-x)}}\exp[-jk_oL_3]\\
&= \frac{1}{4\pi}\int_0^L dx \Gamma \sqrt{\frac{1}{x}}\sqrt{\frac{1}{L-x}}\frac{h_2-s}{L-x}\exp\left[-jk_o\left(L_2+L_3\right) \right]
\label{gf_eq:58}
\end{aligned}
\end{equation}
\renewcommand{\baselinestretch}{2} \small\normalsize

\subsubsection {Knife Edges and Non-planar Terrain}
Following \cite{whitteker_diffraction}, we can formulate the integral in a direction perpendicular to the surface and treat subsequent obstacles as knife edges.

\subsection {Optical Propagation Through Turbulence}

\subsection{Polarization}
In the scalar theory presented here, there is no sense of polarization. This means that once set, the polarization will be constant for all time. Polarization is accounted for through the reflection coefficients and boundary conditions as those implicitly have polarization dependence.
